\chapter{Discussion}
\label{ch:discussion}

This chapter will discuss the results presented in the previous chapter and the ways in which they inform our understanding of Order and alternation in Nêhiywawêwin. The first section will discuss the profiles suggested by the results and how these relate to previous research and the descriptions of Order; the next section will then discuss the statistical veracity of the logistic models. Finally, this chapter will close with a section that discusses and demonstrates how the results of this dissertation can be used to produce exemplars for each alternation outcome, and the ways that this sort of resource can be used in the revitalization of Nêhiywawêwin.

\section{Independent vs. Conjunct}
In the alternation between the Independent and the general Conjunct, the majority of significant effects, regardless of verb class, were predictive of a Conjunct form, rather than the Independent. A general summation of the effects across verb classes in Table \ref{tab:disscivc}. The effects are split into four main categories: effects of actors, effects of goals, effects of preverbs, and effects representing semantic classes. In this last category, conjugation class information is removed from row names (e.g.\codebox{TI.do} and \codebox{AI.do} would be grouped together if both were significant). 


\begin{table}[]
\begin{tabular}{@{}llllll@{}}
\toprule
               & Effects          & II                                                       & AI                                                       & TI                                                       & TA                                                       \\ \midrule
Actor          & actor.obv        &                      & \cellcolor[HTML]{EA9999}{ CNJ}       &                      & \cellcolor[HTML]{EA9999}{ CNJ}       \\
               & actor.sg         &                      & \cellcolor[HTML]{EA9999}{ CNJ}       &                      & \cellcolor[HTML]{B6D7A8}{ CNJ}       \\
               & NA.persons.actor &                      &                                                          & \cellcolor[HTML]{B6D7A8}{ IND}       &                      \\
               & actor.1          &                      &                                                          &                      & \cellcolor[HTML]{B6D7A8}{ IND}       \\
               & actor.2          &                      &                                                          & \cellcolor[HTML]{B6D7A8}{ IND}       &                      \\
               & actor.3          &                      &                                                          & \cellcolor[HTML]{EA9999}{ CNJ}       &                      \\
\midrule
Goal           & NI.Place.goal    &                      &                                            & \cellcolor[HTML]{EA9999}{ CNJ}       &                      \\
               & NDIbody.goal     &                      &                                            & \cellcolor[HTML]{EA9999}{ CNJ}       &                      \\
               & NI.nominal.goal  &                      &                                            & \cellcolor[HTML]{EA9999}{ CNJ}       &                      \\
               & NA.persons.goal  &                      &                                            &                      & \cellcolor[HTML]{EA9999}{ CNJ}       \\
               & goal.obv         &                      &                                            &                      & \cellcolor[HTML]{EA9999}{ CNJ}       \\
               & px1sg.goal       &                      &                                            & \cellcolor[HTML]{B6D7A8}{ IND}       &                      \\
               & goal.2           &                      &                                            &                      & \cellcolor[HTML]{EA9999}{ CNJ}       \\
\midrule
Preverb        & PV.Time          & \cellcolor[HTML]{EA9999}{ CNJ}       & \cellcolor[HTML]{B6D7A8}{ IND}       &                      & \cellcolor[HTML]{EA9999}{ CNJ}       \\
               & PV.Discourse     &                      & \cellcolor[HTML]{EA9999}{ CNJ}       & \cellcolor[HTML]{EA9999}{ CNJ}       &                      \\
               & PV.Position      &                      &                      &                      & \cellcolor[HTML]{EA9999}{ CNJ}       \\
               & PV.Move          &                      &                      &                      & \cellcolor[HTML]{EA9999}{ CNJ}       \\
\midrule
Semantic class & Food             &                      &                      &                      & \cellcolor[HTML]{EA9999}{ CNJ}       \\
               & Do               &                      &                      & \cellcolor[HTML]{EA9999}{ CNJ}       &                      \\
               & Money/count      &                      &                      & \cellcolor[HTML]{EA9999}{ CNJ}       &                      \\ \bottomrule
\end{tabular}
            \caption{
               Multivariate Effects: Independent vs. Conjunct \\ \label{tab:disscivc}
              }
\end{table}


Preverbs seemed to only increase the likelihood of a Conjunct form, with one exception. This behaviour may suggest that the Conjunct is a more modified category. In particular, preverbs of discourse suggest a verb that is not simply declarative in structure, providing some information about the discourse act. This sort of behaviour conforms with the descriptions of \citet[162]{Cook2008}, who purports the Conjunct to be more likely in a medial context; that is, it is more likely to not be at the start of a conversation or story. This description implies the Conjunct order is somehow related to the discourse structure of the utterance. The only preverb associated with the Independent is in the VAIs, where preverbs of time increase the likelihood of the order. This is peculiar for two reasons: Firstly, preverbs of time include \codebox{PV.ka}, an irrealis preverb (usually interpreted as a future definite form in the Independent) that is also present in all ka-Conjunct forms, which make up a large amount of the Other Conjunct class. Following from the first peculiarity, the second is in the disagreement between the VAIs and the VTIs and VTAs in the direction of the effect of \codebox{PV.Time}. Although no single effect was significant in all classes, no effect other than \codebox{PV.Time} differed in its direction of association throughout the verb classes. Actor persons were not significant for all classes, but when present, local actors increased the likelihood of an Independent, while third person actors increased the likelihood of a Conjunct. Also interesting is the distribution of Independent effects across conjugation classes: the VTI had three Independent effects, while the VAI and VTA had only one each. The VII had only one significant effect, that of \codebox{PV.Time}, though this lack of effects is likely due to the lack of tokens in analysis.  Also relating to the Independent, is the fact that Independent effects we almost always argument effects, such as \codebox{actor.1}. A final note, and one that affects all classes, is that \codebox{PV.Position} includes the preverb \textit{ohci-}, which can mean `from', but is also used as a past marker in the Conjunct. As a result, the semantic motivation for this set of preverbs cannot be discerned. Overall, the alternation between the Independent suggests that the Conjunct is a more marked class, and one that is more associated with modifying preverbs, especially of those of time and discourse. 




\section{Independent vs. ê-Conjunct}
The pattern of effects is significantly different for the Independent vs. ê-Conjunct alternation than for the Independent vs. Conjunct alternation. This difference suggests a difference in the \textit{type} of alternation. The effects of this alternation are detailed in Table \ref{tab:disscive}. 

\begin{table}[]
\begin{tabular}{@{}llllll@{}}
\toprule
               & Effects          & II                                                 & AI                                                   & TI                                                   & TA                                                   \\
Actor          & actor.1          &                                                    & \cellcolor[HTML]{B6D7A8}{ IND}   &                                                      & \cellcolor[HTML]{B6D7A8}{ IND}   \\
               & actor.2          &                                                    & \cellcolor[HTML]{B6D7A8}{ IND}   & \cellcolor[HTML]{B6D7A8}{ IND}   & \cellcolor[HTML]{B6D7A8}{ IND}   \\
               & actor.3          &                                                    &                                                      & \cellcolor[HTML]{EA9999}{ ê-CNJ} &                                                      \\
               & actor.obv        &                                                    &                                                      &                                                      & \cellcolor[HTML]{EA9999}{ ê-CNJ} \\
               & NA.persons.actor &                                                    &                                                      & \cellcolor[HTML]{B6D7A8}{ IND}   &                                                      \\
\midrule
Goal           & NI.place.goal    &                                                    &                                                      & \cellcolor[HTML]{EA9999}{ ê-CNJ} &                                                      \\
               & Px1Sg.goal       &                                                    &                                                      & \cellcolor[HTML]{B6D7A8}{ IND}   &                                                      \\
\midrule
Preverb        & PV.Discourse     &                                                    & \cellcolor[HTML]{EA9999}{ ê-CNJ} & \cellcolor[HTML]{EA9999}{ ê-CNJ} & \cellcolor[HTML]{EA9999}{ ê-CNJ} \\
               & PV.Move          &                                                    &                                                      &                                                      & \cellcolor[HTML]{EA9999}{ ê-CNJ} \\
               & PV.Position      &                                                    &                                                      &                                                      & \cellcolor[HTML]{EA9999}{ ê-CNJ} \\
               & PV.Time          & \cellcolor[HTML]{B6D7A8}{ IND} & \cellcolor[HTML]{B6D7A8}{ IND}   &                                                      & \cellcolor[HTML]{EA9999}{ ê-CNJ} \\
\midrule
Semantic class & Do               &                                                    &                                                      & \cellcolor[HTML]{EA9999}{ ê-CNJ} &                                                      \\
               & Food             &                                                    &                                                      &                                                      & \cellcolor[HTML]{EA9999}{ ê-CNJ} \\
               & Money.Count      &                                                    &                                                      & \cellcolor[HTML]{EA9999}{ ê-CNJ} &                                                     \\
               \bottomrule
\end{tabular}
            \caption{
               Multivariate Effects: Independent vs. ê-Conjunct ( \\ \label{tab:disscive}
              }
\end{table}


Similar to the previous alternation, local actors, when significant, always increased the likelihood of an Independent Order, while third person actors increased the likelihood of the ê-Conjunct order for VTIs and obviative actors did the same, but in the VTA class. The presence of an overt actor representing a person also increased the likelihood of the Independent, though only significantly for the VTIs. Together, these effects suggest an Independent order that is associated with higher position in the Nêhiyawêwin person hierarchy (reproduced in (\ref{hier12}), while the Conjunct generally associated positions on the lower level of the hierarchy (i.e. non-local participants).

\begin{exe} % sets up the top-level example environment
\ex\label{hier12} 2 $>$ 1 $>$ Unspecified Actor $>$ 3 $>$ 3^{\prime} $>$ 3^{\prime}^{\prime}
\end{exe}

As previously, the Independent was generally associated with effects dealing with actors and goals, with the exception of prteverbs of time. Unlike the previous alternation, preverbs of time significantly increased the likelihood of the Independent for both the VIIs and the VAIs, though it is still associated with the ê-Conjunct in the VTA class. Other than this set, all other preverb effects which were significant were associated with the ê-Conjunct outcome. It is worth noting, that preverb effects were mostly significant only for the VTA class. In fact, only preverbs of discourse and preverbs of time had effects in any other verb class. This suggests, as in the Independent vs. Conjunct alternation, that the ê-Conjunct is a form that is more marked/altered (except for time). Finally, semantic classes were again only significant when influencing the ê-Conjunct, and even then only in transitive classes. 


\FloatBarrier
\section{Conjunct Type}
The Conjunct Type alternation was significantly less `cohesive'. That is, less can be said about an outcome across verb classes. As can be seen in Table \ref{tab:cnjtypedisc}, even when an effect is present in multiple verb classes, it is not always the case that the effect increased or decreased the likelihood of the same outcome in each conjugation class. (e.g. while \codebox{PV.Discourse} is significant for VAIs and VTAs, the effect increases the likelihood of ê-Conjunct in both, but decreases the likelihood of kâ-Conjunct in the VAIs and Other Conjunct in the VTAs. Similarly \codeboc{PV.Position}, increased the likelihood of the ê-Conjunct in the VAIs, VTIs and VTAs, but it decreases the likelihood of kâ-Conjunct in VAIs and VTAs while decreasing the likelihood of the Other Conjunct in the VTIs.


\begin{table}[h!]
\footnotesize
\begin{tabular}{@{}llllllllllll@{}}
\toprule
               &                     & II                             & \multicolumn{3}{l}{AI}                                                                        & \multicolumn{3}{l}{TI}                                                                             & \multicolumn{3}{l}{TA}                                                                         \\ \midrule
Actor          & actor.1             &                                & \cellcolor[HTML]{B6D7A8}ê-       & \multicolumn{2}{l}{\cellcolor[HTML]{EA9999}kâ-}      &                               &                                &                                   & \cellcolor[HTML]{B6D7A8}ê-      & \multicolumn{2}{l}{\cellcolor[HTML]{EA9999}Other-}     \\
               & actor.2             &                                & \multicolumn{2}{l}{\cellcolor[HTML]{EA9999}ê-}  & \cellcolor[HTML]{B6D7A8}Other-        & \cellcolor[HTML]{EA9999}ê- & \multicolumn{2}{l}{\cellcolor[HTML]{B6D7A8}Other-}              & \multicolumn{2}{l}{\cellcolor[HTML]{EA9999}ê-}  & \cellcolor[HTML]{B6D7A8}Other-         \\
               & actor.3             &                                & \cellcolor[HTML]{B6D7A8}ê-       & \multicolumn{2}{l}{\cellcolor[HTML]{EA9999}kâ-}      &                               &                                &                                   &                                    &               &                                           \\
               & NA.persons.actor    &                                &                                     &              &                                          & \multicolumn{3}{l}{\cellcolor[HTML]{EA9999}kâ-}                                                 &                                    &               &                                           \\
               & NDA.Relations.actor &                                & \multicolumn{2}{l}{\cellcolor[HTML]{B6D7A8}kâ-} & \cellcolor[HTML]{EA9999}Other-        &                               &                                &                                   &                                    &               &                                           \\
               & Pl.actor            &                                & \multicolumn{3}{l}{\cellcolor[HTML]{B6D7A8}kâ-}                                            &                               &                                &                                   &                                    &               &                                           \\
               & Prox.actor          &                                &                                     &              &                                          & \multicolumn{3}{l}{\cellcolor[HTML]{EA9999}ê-}                                                  & \cellcolor[HTML]{EA9999}ê-      & \multicolumn{2}{l}{\cellcolor[HTML]{B6D7A8}kâ-}        \\
               & Sg.actor            &                                & \cellcolor[HTML]{EA9999}ê-       & \multicolumn{2}{l}{\cellcolor[HTML]{B6D7A8}kâ-}      &                               &                                &                                   &                                    &               &                                           \\
               & NI.nominal.goal     &                                &                                     &              &                                          & \multicolumn{2}{l}{\cellcolor[HTML]{EA9999}ê-}              & \cellcolor[HTML]{B6D7A8}Other- &                                    &               &                                           \\
               & NI.object.goal      &                                &                                     &              &                                          & \multicolumn{3}{l}{\cellcolor[HTML]{EA9999}kâ-}                                                 &                                    &               &                                           \\
               & Med.goal            &                                &                                     &              &                                          & \multicolumn{3}{l}{\cellcolor[HTML]{B6D7A8}Other-}                                              &                                    &               &                                           \\
               & Sg.goal             &                                &                                     &              &                                          & \multicolumn{3}{l}{\cellcolor[HTML]{EA9999}ê-}                                                  & \cellcolor[HTML]{EA9999}ê-      & \multicolumn{2}{l}{\cellcolor[HTML]{B6D7A8}kâ-}        \\
               & Prox.goal           &                                &                                     &              &                                          & \multicolumn{3}{l}{\cellcolor[HTML]{EA9999}kâ-}                                                 &                                    &               &                                           \\
               & Px1Sg.goal          &                                &                                     &              &                                          &                               &                                &                                   & \multicolumn{3}{l}{\cellcolor[HTML]{B6D7A8}ê-}                                              \\
Preverb        & PV.Discourse        &                                & \cellcolor[HTML]{B6D7A8}ê-       & \multicolumn{2}{l}{\cellcolor[HTML]{EA9999}kâ-}      &                               &                                &                                   & \multicolumn{2}{l}{\cellcolor[HTML]{B6D7A8}ê-}  & \cellcolor[HTML]{EA9999}Other-         \\
               & PV.Move             &                                &                                     &              &                                          &                               &                                &                                   &                                    &               &                                           \\
               & PV.Position         &                                & \cellcolor[HTML]{B6D7A8}ê-       & \multicolumn{2}{l}{\cellcolor[HTML]{EA9999}kâ-}      & \cellcolor[HTML]{B6D7A8}ê- & \cellcolor[HTML]{B6D7A8}kâ- & \cellcolor[HTML]{EA9999}Other- & \cellcolor[HTML]{B6D7A8}ê-      & \multicolumn{2}{l}{\cellcolor[HTML]{EA9999}kâ-}        \\
               & PV.Qual             &                                & \multicolumn{3}{l}{\cellcolor[HTML]{EA9999}Other-}                                         &                               &                                &                                   &                                    &               &                                           \\
               & PV.WantCan          &                                & \multicolumn{3}{l}{\cellcolor[HTML]{EA9999}kâ-}                                            & \cellcolor[HTML]{EA9999}ê- & \multicolumn{2}{l}{\cellcolor[HTML]{EA9999}kâ-}                 &                                    &               &                                           \\
Semantic class & Cognitive           &                                &                                     &              &                                          &                               &                                &                                   & \multicolumn{2}{l}{\cellcolor[HTML]{EA9999}kâ-} & \cellcolor[HTML]{B6D7A8}Other-         \\
               & Cooking             &                                & \multicolumn{3}{l}{\cellcolor[HTML]{EA9999}kâ-}                                            &                               &                                &                                   &                                    &               &                                           \\
               & Health              &                                & \multicolumn{3}{l}{\cellcolor[HTML]{B6D7A8}Other-}                                         &                               &                                &                                   &                                    &               &                                           \\
               & Speech              &                                &                                     &              &                                          & \cellcolor[HTML]{EA9999}ê- & \multicolumn{2}{l}{\cellcolor[HTML]{EA9999}kâ-}                 &                                    &               &                                           \\
               & Weather             & \cellcolor[HTML]{B6D7A8}kâ- &                                     &              &                                          &                               &                                &                                   &                                    &               &                                           \\
Reduplication  & RdplW               &                                & \multicolumn{3}{l}{\cellcolor[HTML]{B6D7A8}ê-}                                             &                               &                                &                                   &                                    &               &                                           \\ \bottomrule
\end{tabular}
            \caption{
                   Multivariate Effects: Conjunct Type. Each cell is labeled with the outcomes for which an effect is signficant. If a cell is coloured green, the effect increased that outcome, while a red cell represents an effect decreasing likelihood  \\ \label{tab:cnjtypedisc}
              }
\end{table}


In all classes excluding the VII, preverbs of position increased the likelihood of the ê-Conjunct. Conversely, the ê-Conjunct's likelihood was decreased by the presence of second person actors while the Other Conjunct was increased for the same variable. First person actors significantly increased the likelihood of the ê-Conjunct, but only for the AI and TAs. Third person actors also increased the likelihood ê-Çonjunct, but only significantly for the VAIs.

Perhaps most clear is the effect of \codebox{II.weather} on the kâ-Conjunct. The use of a Conjunct form for weather verbs seems to allow for the use of the verb as a durative state as in (\ref{weatherkaa}), where the verb \textit{kâ-pipohk} is used to mean `during winter', rather than being used as a more declarative statement.

\begin{exe}
\ex
\gll ... awâsisak wâwîs \textbf{kâ-pipohk} pîhc-âyih \\
     ... children especially \textbf{in winter} inside \\
\trans `... especially for children in winter.' \citep[137]{Minde1997kwayask}
\label{weatherkaa}
\end{exe}

\FloatBarrier


\section{Model Statistics}
In addition to the actual results of alternation modelling, one can assess the performance of a model as in \citet{arppe2008univariate}. This procedure compares the $\rho^{2}$ and $\tau$ across various models to compare the ability of each model to explain each the three alternations in this dissertation. This section will compare five different models: Fixed Effects-Only (FE) which did not include the random effect \codebox{Lemma}, Random Effects-Only (RE) models which only included \codebox{Lemma} (specifically as a random effect), Semantic-Only (SE) which included the random effect but only semantic variables for fixed effects, Morphological-Only (MorphE) models which had the random effect but only morphological effects for fixed effects, and finally Mixed-Effect (ME) models which include both types of fixed effects and the random effect. While only the ME models were discussed in the Results and this Discussion section, these other models are included in comparison so as to judge the efficacy of the ME model. For example, if a ME model contains a lower $\tau$ and $\rho^{2}$ than, say, an FE or RE model for the same outcome and verb class, it can be determined that use of both fixed and random effects is not advantageous to explaining the alternation, and that a simple generalized linear model would better fit the phenomenon. Similarly, the SE and MorphE models can be compared against each other and the ME to determine the extent to which semantic or morphological information aid in the understanding of the alternations. By default, one would expect the ME models to perform better than either the SE or MorphE (as more information is presumably better for the model), though this may not necessarily be the case. For this dissertation, morphological effects are those which have obvious and easily identifiable morphological exponents, such as \codebox{RdplS} or \codebox{goal.1}. Although some effects are specified for \codebox{actor} or \codebox{goal}, these tags are not considered semantic as they are relatively clearly associated with a suffix or suffix chunk. Below is a list of all morphological effect used in any model:

\begin{lstlisting}[style=mystyle]
actor.1, actor.2, actor.3, actor.4,  D.goal, goal.1, goal.2, goal.3, goal.4, Pl.actor, Pl.goal, Px1Sg.goal, Px3Pl.goal, Px3Sg.goal, RdplW, Sg.actor, Sg.goal\end{lstlisting}

Semantic effects are defined as those which do not have clear morphological exponents. This includes semantic classes, preverb groups, and descriptions of arguments (e.g. \codebox{dem.goal} for goals which are demonstrative. Below is a list of semantic effects used throughout modelling.
  
\begin{lstlisting}[style=mystyle]
AI.cooking, AI.do, AI.health, AI.pray, AI.reflexive, AI.speech, AI.state, Dem.actor, Der.Dim.goal, II.natural.land, II.sense, II.weather, Med.actor, Med.goal, NA.beast.of.burden.actor, NA.beast.of.burden.actor, NA.food.actor, NA.persons.actor, NA.persons.goal, NDA.Relations.actor, NDA.Relations.actor, NDI.Body.goal, NI.natural.force.goal, NI.nominal.goal, NI.object.actor, NI.object.actor, NI.object.goal, NI.place.goal, PV.Discourse, PV.Move, PV.Position, PV.Qual, PV.StartFin, PV.Time, PV.WantCan, PV.e, PV.kaa, PV.kaa, Prox.actor, Prox.goal, Prox.goal, TA.cognitive, TA.do, TA.food, TA.money.count, TA.speech, TI.do, TI.money.count, TI.speech        \end{lstlisting} 
  
Table \ref{tab:randvsfullivc} details the difference in $\tau$ and $\rho^{2}$ between the five different types of models previously described for the Independent vs. Conjunct alternaiton.
  
        \begin{table}[h!]
            \centering
            \begin{tabular}{lllllllll}
            \toprule
            & \multicolumn{2}{c}{VII}                          & \multicolumn{2}{c}{VAI}                          & \multicolumn{2}{c}{VTI}          & \multicolumn{2}{c}{VTA}                         \\
            & $\tau$      &$\rho^{2}$      & $\tau$      &$\rho^{2}$ & $\tau$           &$\rho^{2}$      & $\tau$      &$\rho^{2}$ \\
            \midrule
            Fixed Effect-Only  & 0.31 & 0.05 & 0.47 & 0.15          & 0.32 & 0.09 & 0.35 &  0.10 \\
            Random Effect-Only & 0.36 & 0.12 & 0.52 & \textbf{0.26} & 0.34 & 0.15 & 0.42 & 0.18 \\
            Semantic-Only      & 0.35 & 0.12 & 0.52 & \textbf{0.27} & 0.36 & 0.15 & 0.44 & 0.19 \\
            Morphological-Only &      &       &0.52 & \textbf{0.27} & 0.39 & 0.16 & 0.43 & 0.19 \\
            Mixed Effects      & 0.35 & 0.12 & 0.53 & \textbf{0.27} & 0.42 & 0.16 & 0.45 & \textbf{0.21} \\


            \bottomrule
            \end{tabular}
            \caption{
               Model Comparisons. Independent vs. Conjunct, bold items represent a very good model fit, per \citet{mcfadden1973conditional} \\ \label{tab:randvsfullivc}
              }
        \end{table}


In this alternation, it was nearly always the case that ME models had superior performance in both classification improvement ($\tau$) as well as reduction in badness of fit ($\rho^{2}$). This is not the case in two instances: the first is in the VII class, where RE models appeared to have slightly higher $\tau$ and $\rho^{2}$ than the ME model, despite containing less information in terms of predictor. It is worth noting that the VII were the least numerous class, and due to its inherent semantics is significantly different than the other classes (in that it can refer to things like days of the week or temporal states). Also worth mentioning is that MorphE models were not available for the VII class, as the significant fixed effects for VIIs were all semantic. The other case is in the VTIs where the $\rho^{2}$ was the same for ME models as the MorphE models. In all cases, FE models showed significantly lower measures the other models, with RE models showing a significant increase in $\rho^{2}$ and $\tau$ measures over FE models. This indicates that a significant amount of alternation is explained by random effects only. Put another way, individual lemmata appear to show significant variation in their propensity to occur in the Independent or Conjunct order more so than the use of fixed effects alone. The SE and MorphE models create a slight increase in $\rho^{2}$ and $\tau$ over the RE models,. While SE and MorphE vary in which produces a better model depending on the verb conjugation class, this difference is usually minimal. As mentioned before, the ME often better than any other model, and excluding the VIIs, they are never \textit{worse} than the other models in their ability to classify or reduce badness of fit. 

       
        \begin{table}[h!]
            \centering
            \begin{tabular}{lllllllll}
            \toprule
            & \multicolumn{2}{c}{VII}                          & \multicolumn{2}{c}{VAI}                          & \multicolumn{2}{c}{VTI}          & \multicolumn{2}{c}{VTA}                         \\
            & $\tau$      &$\rho^{2}$      & $\tau$      &$\rho^{2}$ & $\tau$           &$\rho^{2}$      & $\tau$      &$\rho^{2}$ \\
            \midrule
            Fixed Effect-Only  & 0.28 & 0.04 & 0.43 & 0.15 & 0.33 & 0.10 & 0.32 & 0.12 \\   
            Random Effect-Only & 0.35 & 0.14 & 0.49 & \textbf{0.26} & 0.36 & 0.17 & 0.38 & 0.17 \\ 
            Semantic-Only      & 0.40 & 0.17 & 0.50 & \textbf{0.27} & 0.38 & 0.17 & 0.41 & 0.19 \\
            Morphological-Only &      &      & 0.50 & \textbf{0.26} & 0.41 & 0.19 & 0.44 & \textbf{0.21} \\
            Mixed Effects      & 0.40 & 0.17 & 0.51 & \textbf{0.27} & 0.44 & \textbf{0.20} & 0.43 &\textbf{0.22} \\
        
            \bottomrule
            \end{tabular}
            \caption{
               Model Comparisons. Independent vs. ê-Conjunct \\ \label{tab:randvsfullive}
              }
        \end{table}


For the Independent vs ê-Conjunct alternation, a similar pattern to the previous alternation can be seen. In general, FE models provided very little explanation for the variation, while RE models showed a significant higher $\rho^{2}$. This again suggests that lemmas have inherent propensities to surface in one order over another. SE and ME models showed similar reduction in badness of fit and classification efficacy. In all cases, ME models were the best fitting models, except for the VIIs where it was equal to the MorphE model. Similarly, all classes other than the VII showed had ME models with an $\rho^{2} \geq$ 0.20. The lower measure for the VII model is again likely the result of a paucity of data. Taken together, the model performance for the Independent vs. ê-Conjunct alternation suggest a generally better modeled alternation than in the Independent vs. Conjunct general alternation. Although theoretically one may conceive of Order as a class that is split principally between the Independent and Conjunct (as in Chapter \ref{ch:order}, these results suggest that such an alternation is harder to model in terms of morpho-semantic properties. Instead, a more clear choice exists in whether one wants to use an Independent form or an ê-Conjunct form. This behaviour is not entirely unexpected, given the semantic differences of the different Conjunct types. Even \citet{Cook2014} describes the ê-Conjunct as more of an elsewhere case and one more closely similar to the Independent in morphosyntactic behaviour. In fact, \citet[125]{Cook2014} describes all conjunct type other then the ê-Conjunct as being disallowed from matrix clauses, which are the domain of the Independent and the ê-Conjunct. This comports with the performance of the above models. Further, the Other Conjuncts are highly semantically representative in a way that does not need accounting for (and was not accounted for) in the models, as will be discussed below. Thus, mixing the ê-Conjunct together with these other, more straightforwardly described forms as in the first alternation, may produce an outcome that a logistic model is simply not fully able to reproduce and explain. 


\begin{table*}
\begin{minipage}[t]{\columnwidth}
\caption{Model Comparisson. Conjunct Type Alternation, ê-Conjunct outcome} \label{tab:modcompcnjtype}
\centering 
        \begin{tabular}{lllllllll}
            \toprule
        \textsc{ê-Cnj}    & \multicolumn{2}{c}{VII}                          & \multicolumn{2}{c}{VAI}                          & \multicolumn{2}{c}{VTI}          & \multicolumn{2}{c}{VTA}                         \\
            & $\tau$      &$\rho^{2}$      & $\tau$      &$\rho^{2}$ & $\tau$           &$\rho^{2}$      & $\tau$      &$\rho^{2}$ \\
            \midrule
            Fixed Effect-Only  & 0.31 & 0.11 & 0.34 & 0.04 & 0.33 & 0.07 & 0.31 & 0.06   \\    
            Random Effect-Only & 0.56 & \textbf{0.33} & 0.34 & 0.14 & 0.36 & 0.18 & 0.33 & 0.11  \\    
            Semantic-Only      & 0.57 & \textbf{0.33} & 0.35 & 0.15 & 0.41 & \textbf{0.20} & 0.35 & 0.13  \\
            Morphological-Only &  &  & 0.39 & 0.16 & 0.40 & \textbf{0.20} & 0.38 & 0.15 \\
            Mixed Effects      & 0.57 & \textbf{0.33} & 0.39 & 0.17 & 0.44 & \textbf{0.22} & 0.40 & 0.17 \\
        
            \bottomrule
            \end{tabular}
\end{minipage}\hfill
\begin{minipage}[t]{\columnwidth}
%\caption{Model Comparisson. Conjunct Type Alternation, kâ-Conjunct outcome} \label{tab:modcompcnjtypekaa}
\centering 
        \begin{tabular}{lllllllll}
            \toprule
        \textsc{kâ-Cnj}    & \multicolumn{2}{c}{VII}                          & \multicolumn{2}{c}{VAI}                          & \multicolumn{2}{c}{VTI}          & \multicolumn{2}{c}{VTA}                         \\
            & $\tau$      &$\rho^{2}$      & $\tau$      &$\rho^{2}$ & $\tau$           &$\rho^{2}$      & $\tau$      &$\rho^{2}$ \\
            \midrule
            Fixed Effect-Only  & 0.31 & 0.13 & 0.38 & 0.03 & 0.42 & 0.06 & 0.38 & 0.03   \\    
            Random Effect-Only & 0.48 & \textbf{0.33} & 0.46 & 0.14 & 0.45 & 0.16 & 0.45 & 0.11  \\    
            Semantic-Only      & 0.54 & \textbf{0.33} & 0.40 & 0.15 & 0.45 & 0.19 & 0.38 & 0.12  \\
            Morphological-Only &  &  & 0.40 & 0.15 & 0.43 & 0.17 & 0.38 & 0.13 \\
            Mixed Effects      & 0.54 & \textbf{0.33} & 0.40 & 0.16 & 0.44 & 0.19 & 0.39 & 0.13 \\
        
            \bottomrule
            \end{tabular}
\end{minipage}
\begin{minipage}[t]{\columnwidth}
%    \caption{Model Comparisson. Conjunct Type Alternation, Other Conjunct outcome} \label{tab:modcompcnjtypeother}
    \centering 
            \begin{tabular}{lllllllll}
                \toprule
                \textsc{Other Cnj} & \multicolumn{2}{c}{VII}                          & \multicolumn{2}{c}{VAI}                          & \multicolumn{2}{c}{VTI}          & \multicolumn{2}{c}{VTA}                         \\
                & $\tau$      &$\rho^{2}$      & $\tau$      &$\rho^{2}$ & $\tau$           &$\rho^{2}$      & $\tau$      &$\rho^{2}$ \\
                \midrule
                Fixed Effect-Only  & 0.48 & 0.02 & 0.47 & 0.08 & 0.43 & 0.07 & 0.48 & 0.10   \\    
                Random Effect-Only & 0.48 & \textbf{0.24} & 0.45 & 0.17 & 0.42 & 0.19 & 0.43 & 0.17  \\    
                Semantic-Only      & 0.48 & \textbf{0.33} & 0.48 & \textbf{0.26} & 0.45 & \textbf{0.25} & 0.45 & \textbf{0.22} \\
                Morphological-Only &   &   & 0.50 & \textbf{0.26} & 0.46 & \textbf{0.26} & 0.51 & \textbf{0.26} \\
                Mixed Effects      & 0.48 & \textbf{0.33} & 0.51 & \textbf{0.27} & 0.47 & \textbf{0.28} & 0.52 &\textbf{0.27} \\
            
                \bottomrule
                \end{tabular}
    \end{minipage}
\end{table*}


    In Conjunct type alternation, as depicted in Table \ref{tab:modcompcnjtype}, models performed variously well depending on the outcome. In all cases, the pattern of the FE models performing poorly with more morphological and semantic information producing better models continues. Despite this, only in the  VII RE, SE, and ME and the VTI SE, MorphE, and ME models were $\rho^{2} \geq$ 0.20. In general then, modelling was not particularly successful for this outcome. The kâ-Conjunct outcome fairs even worse, with only the VII RE, SE, and ME model showing very good fits. Interestingly, the kâ-Conjunct model showed a large increase in $\rho^{2}$ from FE to RE models, but SE, MorphE, and ME models showed only minor increases, suggesting that semantic and morphological information do not do much to determine the propensity of a lemma to take the kâ-Conjunct outcome. Instead, the largest amount of variance explanation is found in lemma specific effects, as seen in the very large increase of $\rho^{2}$ from FE to RE. The Other Conjunct differed from the other two outcomes in that all SE, MorphE, and ME models were very well fit (as was the RE model for VIIs). While RE models were significantly better fit than FEs as previously, an additional large jump in $\rho^{2}$ value is seen when adding semantic or morphological information to the model. The use of both type of effects in the ME model provides a slight increase over other models. These results seem very promising, but when actually evaluating the models ability to predict the correct outcome, however, an interesting pattern emerges. While recall for the alternate outcome is roughly 100\% (that is, the model is correctly predicting all forms which do not occur in the Other Conjunct), recall for the Other Conjunct outcome is never higher than 20\% for any class. Despite this, precision for both outcomes in all classes in relatively high, especially for the VAIs and VTAs (where it is higher than 80\%). This is the behaviour of a model that predominantly predicts one outcome regardless of contextual information, almost never deviating from this pattern. Because the vast majority of training tokens were, in fact, part of the alternate outcome (and not the Other Conjunct), this behaviour produces a technically well fit model, despite not actually being particularly more insightful than frequency counts. In total, the Conjunct Type alternation is a poorly fit alternation when compared with the Independent vs. Conjunct and Independent vs. ê-Conjunct alternations. This is not entirely surprising considering the nature of the alternation. While the Independent vs. Conjunct and Independent vs. ê-Conjunct are alternations between two outcomes that vary based on some unknown factor, the Conjunct Type outcomes are highly motivated by the semantics of the utterance. For example, if a verb is in a relative clause such as  \textit{awîniki aniki \textbf{kâ-mêtawêcik} anita?}, `Who are \textbf{those that are playing there}' \citep[46]{okimasis2018cree}, it will occur in a kâ-Conjunct form; if the verb is used for hypotheticality as in \textit{\textbf{mîcisoyani} kika-miyomahcihon}, `If you eat, you feel better' \citep[64]{okimasis2018cree}, then it will occur in a subjunctive form (and thus an Other Conjunct outcome); etc. Because of this relatively straightforwardness, it is unsurprising that the morphosemantic effects used in this dissertation (which are used to model less straightforward variance) would not apply to this alternation.


Speaking more generally about the fitting of the models in this dissertation, in contrast with \citet{arppe2008univariate} who concludes that while morphosyntactic effects provide some amount predictive power, it is the semantic effects that are most predictive, this dissertation finds that semantic and morphological effects are similarly effective as predictors. This varied by alternation type and conjugation class, but overall the difference between semantic and morphological effects was slight. The reason for the discrepancy between these results and those of \citet{arppe2008univariate} could be due to differences in the ways semantic classes were defined, or it could be due to differences between the languages being studied. While Finnish is a synthetic language with a rich case system, its verbal system is not as morphological complex, lacking a wide system of preverbs or polypersonal agreement. It is possible that Nêhiyawêwin's polysynthesis bolsters the explanatory power for morphological effects in modelling these alternations. Additionally, the nature of the alternations being studied in \citet{arppe2008univariate} are significantly different than those in this dissertation. While \citet{arppe2008univariate} focused on an alternation between near synonyms generally to do with a single semantic domain (\textit{thinking)}, the alternation of Order is more structural, as described in Chapter \ref{ch:order}. Finally, the research in the dissertation differ from \citet{arppe2008univariate} methodologically in that the models used in this research make is of mixed effects models. Although it is not immediately obvious how controlling for the random effect of lemma identity would affect the usefullness of semantic and morphological predictors, its inclusion is significant. Despite the differences between \citet{arppe2008univariate} and this research, both studies demonstrate the ability to model alternations in very divergent languages using logistic modelling.

\section{Exemplar Extraction}
Exemplar extraction was mostly successful, and useful exemplars were presented in Chapter \ref{ch:result}. Interestingly, in actually clustering tokens in order to extract exemplars, a pattern emerged. As a reminder, the purpose of clustering tokens was to avoid over-representation of a few items which all contained a very predictive variable . Despite the clustering work done, in some cases this still occurred. For example, nearly all properly predicted (and with high estimated probabilities) Independent VAIs, regardless of which alternation they were seen in, were a form of the lemma \textit{itwêw}. Although one could simply select only one such exemplar for any given lemma,  there are a number of isssues in doing so. Most importantly, does so ignores the fact that it is a fact that the estimated probabilities for \textit{itwêw} are significant higher than other lemmas (e.g., in the Independent vs. Conjunct alternation, the lowest estimation for a \textit{itwêw} lemma is 73\%, while the next highest rated token of a different lemme is a full 10 percent-points lower). By choosing to ignore all but one \textit{itwêw} lemma, one would  be highlighting less certain predictions. Further, it is also worth highlighting the fact that VAI quotatives are closely related to the Independent. A second pattern was the fact that all tokens of a given cluster were predicted to occur from one outcome or another, rather than tokens predicted for both outcomes beings present in a single cluster. This makes sense, given that the clustering was based on dataframes based off only significant variables for the alternation prediction, including the predicted varied (e.g. the presence of a \codebox{PV.e} variable in a model predicting if a lemma occurs as an ê-Conjunct).