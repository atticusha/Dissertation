\chapter{Introduction}
\label{ch:intro}

This dissertation explores the phenomenon of Nêhiywêwin, also known as Plains Cree, Order. Nêhiywêwin, like all Algonquian languages, is a polysynthetic language with rich verb morphology. The most striking system in Algonquian verbs morphology, apart from perhaps the language's hierarchical/direct-inverse alignment \citep[53]{Wolvengrey2011}, is the system of Order. Order is instantiated on verbs through a system of allomorphy of the polypersonal argument morphs. While other Algonquian languages differ in their number of Orders, Nêhiywêwin has three recognized Orders: the Imperative, the Independent, and the Conjunct. The function of Order has been discussed in the literature, most comprehensively by \citet{Wolfart1973} and \citet{Cook2014}. The latter resource is most comprehensive, though it generally focuses only on the difference between the Independent and the Conjunct Orders, placing aside the Imperative Order. This is, in my opinion, valid, but unmotivated in \cite[11]{Cook2014} who justifies their decision as `There is a third paradigm: the imperative order. The imperative order cannot host most agreement, any of the elements on the far left edge, or most of the preverbs. I will not discuss it further.' Put simply, \citet{Cook2014} proposes that the phenomenon of Order is one of clause typing, specifically in the difference between Indexical (not having a prior referrent) and Anaphoric (having a prior refferent) clauses. This conclusion was come to after careful hand-analysis of a Nêhiyawêwin corpus. Although I agree with many of the conclusions put forth by \citet{Cook2014}, the orientation of this research is decidedly theoretical. 

This dissertation will approach the purpose and function on Nêhiyawêwin from a systematic and empirical perspective. Using a corpus (including, in part, all of the texts used by \citet{Cook2014}) and modern computational techniques, this dissertation attempts to uncover the motivations in the choice to use one Order over another. This research is undertaken through the lens of \textit{alternation}. The primary method of analysis this dissertation relies on is mixed-effects logistic regression, based on the work of \citet{arppe2008univariate}. By framing Order as a system of alternation, mixed-effects logistic regression allows for the creation of a predictive model, where each of the predictor variables can be evaluated for their effect on the outcome of the alternation. Three types of alternations are investigated: Independent vs. Conjunct (the most straightforward alternation in terms of previous description of Order), Independent vs. ê-Conjunct (the most straightforward alternation in term of near-synonymy \citep[157-159]{cruse2001}), and the alternation of the various Conjunct types (a more straightforwardly semantic alternation).

Chapter \ref{ch:order} provides a background on Nêhiyawêwin, Order, and the use of alternation in linguistic investigation. This chapter also provides a detailed discussion regarding the nature of Order as an alternation, how this outlook can be used to study the phenomenon, and a detailed justification for the ignoring of the Imperative mood beyond methodological opportunism. 

Next, Chapter \ref{ch:semantics} presents a study in semi-automatically clustering verbs together for the purposes of predictor generation for the logistic modelling at the centre of this dissertation. This chapture focuses on how one can use pre-existing majority language data to bootstrap the creation of an ontology for lemmas in a minority language, Nêhiyawêwin. The result of this research, a semantic class for every verb in a dictionary \citep{Wolvengrey2001}, was used as the main semantic effects in the main statistical modeling of this dissertation. Earlier versions of this research were published in \cite{harrigan-arppe-2021-leveraging} and \cite{harriganPACsem}.

Following the chapter of semantic classification, Chapter \ref{ch:method} describes and justifies the particular methodologies in statistical modelling. This chapter also details the morphosyntactically tagged corpus that is being used, and the ways in which this corpus has been construed as a data set. The main research questions driving this analysis are: 

\begin{enumerate}
    \item Can mixed effects modelling be used in investigating complex morphological phenomenon using a small but robust corpus?
    \item Are the alternations between the Independent and Conjunct, the Independent and the ê-Conjunct, and the Conjunct types similarly able to be modelled, or are some of these alternations easier to model than others?
    \item What are the variables that increase the likelihood that a lemma will occur in a particular Order/outcome?
\end{enumerate}

Chapter \label{ch:result} presents the results the statistical modelling in three stages: univariate, bivariate, and multivariate. As the latter is of primary interest for this dissertation, it is discussed most in depth. The following chapter, Chapter \label{ch:disscussion}, discusses in depth the multivariate results. This includes not only a discussion of what this means in the general sense of Order as well as how the results frame each outcome, but also how well the statistical modelling performed and what this overall success or failure can tell us about alternations and Order more generally.