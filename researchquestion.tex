\chapter{Research Question}
\label{ch:rq}
\doublespace
The behaviour of Nêhiyawêwin Orders and modes is fundamentally one of a linguistic alternation. Linguistic alternations are at the core of many linguistic pursuits. Beyond the concepts of allophony and allomorphy, which fundamentally concern the alternation of sound and form-meaning pairs (respectively),  linguistic alternation comes mainly in three forms which correspond to the levels of representation and allow us to account for differences in meaning between nearly-synonymous words and constructions: conceptual-semantic alternation, stylistic-semantic alternation, and a syntactic-semantic alternation (\citealt[8]{arppe2008univariate};  cf. \citealt{edmonds2002near} for an earlier discussion of a similar concept). According to \citet[8]{arppe2008univariate}, conceptual-semantic alternations concern words that mean generally the same thing and can be used (roughly) interchangeably (e.g. \textit{dash} and \textit{sprint}); Stylistic-semantic alternations occur between words or phrases that share similar meanings, but contain different connotations (\textit{poop} and \textit{shit}); and syntactic-semantic alternations deal with similar utterances which take different syntactic patterns (\textit{comb (through)} and \textit{inspect}). These forms of representation roughly correspond to those of Hanks’ lexical, semantic, and syntactic-type alternations (\citeyear[173]{hanks2013lexical}). \citet[10]{arppe2008univariate} also proposes a subset of syntactic-semantic alternations  referred to as constructional alternations,

\begin{quote}
    ``[which] resemble lexical synonymy in that the essential associated meaning is understood to remain for the most part constant regardless of which of the alternative constructions is selected, though they may differ with respect to e.g. some pragmatic aspect such as focus.” 
    \end{quote}
    
    The differences between the different orders and modes of Nêhiyawêwin are proposed to show exactly this difference: a verb in the Independent or Conjunct Orders can describe the same experienced phenomenon. Certainly, the alternation is not simply one of morphological or phonological alternation: there seems to be subtle differences in meaning between Orders and modes.

Viewed as a an analysis of an alternation, the primary research question of this dissertation is as follows: what morphosyntactic and semantic contextual features affect a lemma’s propensity to occur in a particular alternation of Order or mode. Adopting a usage based approach based in the distributional hypothesis \citep{harris1954distributional, firth1957synopsis}, this research will utilize quantitative methodologies in an effort to see to what extent empirical, corpus-based evidence can guide us in the understanding of Nêhiyawêwin order. 
