The composition of the Imperative Order suffixes are similar on the surface to the Conjunct Order's, as mentioned by \citet{Wolfart1973}, though this similarity is likely a historical coincidence, rather than any sort of genuine alternation. This similarity is most clear in the VTA paradigm, as seen in Table \ref{tab:vtaimp}. Here, bolded forms show instances where the difference between a Conjunct and differ only in the Conjunct containing a $<$y$>$, the Delayed Imperative containing a $<$hk$>$, and the conjunct's inclusion of the \{ê-\} prefix.



\begin{table}
  \centering
  \footnotesize
  \begin{tabular}{lll}
    \toprule
     & Conjunct & Delayed Imperative           \\
    \midrule
2\textsc{sg}$\rightarrow$1\textsc{sg} & ê-mowi\textbf{y}an (`you eat me') & mowî\textbf{hk}an (`eat me later!')\\
2\textsc{sg}$\rightarrow$1\textsc{pl} & ê-mowi\textbf{y}ahk (`you eat us')  & mowî\textbf{hk}âhk (`eat us later!') \\
2\textsc{sg}$\rightarrow$3\textsc{sg} & ê-mowat (`you eat him/her') & mowâ\textbf{hk}an (`eat him/her later!') \\
2\textsc{sg}$\rightarrow$3\textsc{pl} & ê-mowacik (`you eat them') & mowâ\textbf{hk}anik (`eat them later!') \\
21\textsc{pl}$\rightarrow$3\textsc{sg} & ê-mowa\textbf{y}ahk (`we all, eat him/her') & mowâ\textbf{hk}ahk (`let us all, eat him/her later!')\\
21\textsc{pl}$\rightarrow$3\textsc{pl} & ê-mowa\textbf{y}ahkok (`we all eat them') & mowâ\textbf{hk}ahkok (`let us all, eat them later!') \\
2\textsc{pl}$\rightarrow$1\textsc{sg}  & ê-mowi\textbf{y}êk (`you all eat me') & mowî\textbf{hk}êk (`you all, eat me later!') \\
2\textsc{pl}$\rightarrow$1\textsc{pl}  & ê-mowi\textbf{y}âhk (`you all eat us')& mowî\textbf{hk}âhk (`you all, eat us later!')\\
2\textsc{pl}$\rightarrow$3\textsc{sg}  & ê-mowa\textbf{y}êk (`you all eat him/her') & mowâ\textbf{hk}êk (`you all, eat him/her later!') \\
2\textsc{pl}$\rightarrow$3\textsc{pl}  & ê-mowa\textbf{y}akok (`you all eat them') & mowâ\textbf{hk}êkok (`you all, eat them later!') \\
    \bottomrule
  \end{tabular}
  \caption{
    VTA Conjunct and Delayed Imperative Paradigm.\label{tab:vtaimp}
  }
\end{table}



Although this surface level similarity may seem suggestive of some sort of Conjunct-Delayed Imperative relationship, not all forms demonstrate the correspondence of $<$y$>$ and $<$hk$>$, for example in Second person singular acting on third person (singular and plural). In this way, one may be tempted to classify the Delayed mode of the Imperative as a set of Conjunct Order forms. Although this is a structurally enticing option (especially for forms including only speech act participants), it is significantly flawed for a number of reasons. Most prominently, the $<$y$>$ and $<$hk$>$ segments are not of the same type. The $<$y$>$ formative is, originally, an epenthetic character used to join conjucnt endings to vowel final stems. Although it is often reanalyzed by contemporary speakers are part of the portmanteau verbal person markers, there is no evidence to suggest that it is thought of as a morpheme in and of itself by any substantial number of speakers. This is contrasted with $<$hk$>$, which was historically the marker of a delayed imperative in and of itself. Even from a synchronic analysis, which is one favoured in this dissertation, this similarity is best explained as the Delayed Imperative simply using similar, though not identical, person and number endings as the Conjunct. 

That  one would expect the Delayed Imperative forms to be able to take the morphological modes of the Conjunct as described above, just as all other Conjunct allow for. This is not something that is described in any literature, and it is difficult to imagine what a Subjunctive Delayed Imperative would actually mean (as the Subjunctive represents a hypothetical proposition in Nêhiyawêwin and it is unclear what a hypothetical command might be); further, if we analyze the Imperative Order as something primarily encoding mood,\footnote{Due to the various interchangeable uses of the terms \textit{mood} and \textit{mode} throughout linguistic literature, I will be exclusively using the term \textit{mood} to discuss the linguistic items which indicate a speaker's attitude toward a statement, including (ir)realisness, ability, necessity, etc. Where the term \textit{mode} is used, it refers to a subtype of a conjunct form.} it would seem unlikely that one can add further moods such as hypotheticality to an Imperative Order form. 

\begin{table}
  \centering
  \footnotesize
\begin{tabular}{lllll}
    \toprule
     & Independent    & Immediate Imperative & Conjunct      & Delayed Imperative        \\
    \midrule
2\textsc{sg}  & kiwapahtên     & wâpahta        & ê-wâpahtaman  & wâpahtam\textbf{ôhk}an  \\
21\textsc{pl} & kiwâpahtênânaw & wâpahtêtân     & ê-wâpahtamahk & wâpahtam\textbf{ôhk}ahk \\
2\textsc{pl}  & kiwâpahtênâwâw & wâpahtamok     & ê-wâpahtamêhk & wâpahtam\textbf{ôhk}êk        \\
    \bottomrule
  \end{tabular}
  \caption{
    Sample of VTI Paradigm Comparing the Independent, Conjunct, and the Imperative. \label{tab:vtiimp2}
  }
\end{table}

Algonquian languages more closely related to Nêhiyawêwin show a similar pattern. Nishnaabemwin/Ojibwe in particular had Imperatives that run parallel to Nêhiyawêwin: 


\begin{table}[h]
  \centering
  \footnotesize
  \begin{tabular}{lll}
    \toprule
     & Conjunct & Delayed Imperative           \\
    \midrule
2\textsc{sg}  & boodwe\textbf{y}in  (`You make fire!')     & boodwe\textbf{k}an (`make fire later!')\\
21\textsc{pl} & boodwe\textbf{y}ing (`We all make fire!') & \\
2\textsc{pl}  & boodwe\textbf{y}eg  (`You all make fire!')  & boodwe\textbf{k}eg (`make fire later!')\\
    \bottomrule
  \end{tabular}
  \caption{
    Nishnaabemwin VAI Conjunct and Delayed Imperative Paradigm.\citep[238]{valentine2001}\label{tab:nishvai}
  }
\end{table}

Although the second person inclusive is not a possible for in Nishnaabemwin, the general pattern of the Conjunct {-y-} being replaced with {-k-} is functionally similar to what has been described in Nêhiywêwin, seen in Table \ref{tab:nishvai}. More disparate Algonquian languages, such as Innu-aimun, also share this pattern.  As seen in Table \ref{tab:eastcree}. 

\begin{table}
  \centering
  \footnotesize
  \begin{tabular}{lll}
    \toprule
     & Conjunct & Delayed Imperative           \\
    \midrule
2\textsc{sg}  & e nipaayan  (`You sleep')     & nipaahkan  (`You sleep later!') \\
21\textsc{pl} & e nipaayahkw (`We all sleep') &    \\
2\textsc{pl}  & e nipaayekw  (`You all sleep')  & nipaawaahkan  (`You all sleep!')\\
    \bottomrule
  \end{tabular}
  \caption{
    Innu-aimun VAI Conjunct and Delayed Imperative Paradigm (https://southern.verbs.eastcree.org/?17a).\label{tab:eastcree}
  }
\end{table}
 
 Even more distantly related Algonquian languages do not even have a delayed imperative. These differences across languages and the lack of an account for a complete Conjunct and Delayed Imperative in Proto-Algonquian make it difficult to determine any sort of historical cause for the similarities between the Conjunct and the Delayed Imperative beyond the dialect Continuum. It seems possible then, though speculative, that the Delayed Imperative is a recently innovated form that is built from the Conjunct word endings. Finally, although Imperative endings resemble the Conjunct endings of the same person, they \textit{cannot} take conjunct preverbs such as \{ê-\} or \{kâ-\}. As a result, I will not consider the Delayed Imperative as a form of Conjunct. 