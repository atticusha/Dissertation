	\chapter{Background}
	\label{ch:order}
Nêhiyawêwin ( ᓀᐦᐃᔭᐍᐏᐣ \textsc{ISO 639-3: crk}, also known as Plains Cree) is a member of the Algonquian language family, one of the major language families of North America, spoken across much of Canada and the eastern United States. Nêhiyawêwin is part of two major dialect continua: the Cree-Montagnais-Naskapi languages, which stretch mainly from Alberta to Quebec and Labrador, and the central Canadian Ojibwe-Cree continuum, which occur in central Canada and the Northern United States. \footnote{The languages/dialects of Cree-Montagnais-Naskapi continuum share many similarities, as can be seen in the following words for `person', where the reflexes of Proto-Algonquian *\textit{l} are given in boldface. The dialects are given in roughly west to east geographical distribution:

\begin{exe}
\ex
Plains: \textit{i\textbf{y}iniw}\\
Woods: \textit{i\textbf{th}iniw}\\
Swampy: \textit{i\textbf{n}iniw}\\
Moose: \textit{i\textbf{l}iliw}\\
Atikamekw: \textit{i\textbf{r}iniw}\\
East: \textit{i\textbf{y}iyiw/i\textbf{y}iyû/i\textbf{y}inû}\\
Naskapi: \textit{i\textbf{y}iyû}\\
Innu: \textit{i\textbf{l}nu/i\textbf{n}nu}
\end{exe}
}

Nêhiyawêwin is the westernmost member of the Cree-Naskapi-Montagnais continuum and is mostly spoken in Alberta, Saskatchewan, and northern Montana. There is said to be approximately 34,000 speakers of Nêhiyawêwin \citep{Ethnologue2016}, most over the age of 30. This number is likely overestimated, though a previous account by Ethnologue was dubious, with a number of roughly 150. Statistics Canada (\citeyear{StatsCan}) reports 3,655 native speakers of `Plains Cree,' though this number may be higher if respondents reported their native language as `Cree,' rather than `Plains Cree.' Wolfart (\citeyear{Wolfart1973}) estimated 20,000 speakers, though the number has likely dropped since then. Although any of these numbers is dwarfed by the number of speakers of majority languages in Canada, Nêhiyawêwin retains a strong presence, particularly for an Indigenous North American language, holding a classification of 5 (Developing) on the Extended Graded Intergenerational Disruption Scale (EGIDS) \citep{Ethnologue2016}, a system for assessing language vitality based on domains of use, intergenerational transmission, and other sociolinguistic factors \citep{LewisSimons2010}. With its comparatively large speaker base, Nêhiyawêwin has garnered attention from a variety of Americanists, in the form of grammars (e.g. \citeauthor{Wolfart1973} \citeyear{Wolfart1973}, \citeauthor{Dahlstrom2014} \citeyear{Dahlstrom2014}, \citeauthor{Wolvengrey2011} \citeyear{Wolvengrey2011}), textbooks (e.g. \citeauthor{okimasis2018cree} \citeyear{okimasis2018cree}; \citeauthor{Ratt2016} \citeyear{Ratt2016}) and an online electronic dictionary (itw\^{e}wina\footnote{\texttt{http:/itwewina.altlab.app}}).

\section{Nouns}

Nêhiyawêwin exhibits a number of morphosyntactic features that differ considerably from the well-known characteristics of often discussed Indo-European languages. Unlike sex-based gender systems such as those found in many contemporary romance languages, Algonquian languages have a two-way gender or noun classification system contrasting inanimate with Animate nouns; this grammatical animacy has some basis in semantic animacy: all humans, animals, and trees are animate. This distinction is not clear-cut though, as \textit{\^{e}mihkw\^{a}n}, `spoon', \textit{sîwinikan}, `sugar', and \textit{sêhkêpayîs}, `automobile' are animate,\footnote{It is worth noting that animacy is not always consistent across dialects of Cree, or even communities of Nêhiyawêwin. Some words, such as \textit{s\^{i}winikan} `sugar', are animate in some dialects and inanimate in others.} and thus the system is considered one of grammatical classification. Notably there are few-to-no examples of clearly alive animals that are grammatically inanimate. Animacy is relevant to nominal and verbal morphology in Nêhiyawêwin in various ways. Among nouns, this animacy distinction is manifested in two distinct plural markers, \{-ak\} for animate and \{-a\} for Inanimate nouns; archaic singular marking is seen for monosyllabic roots, for example \textit{maskw-a} `bear (\textsc{anim})' and \textit{w\^{a}w-i} `egg (\textsc{inanim})'. Nêhiyawêwin has no grammatical case system, but it does have locative marking, generally -\textit{ihk} for Inanimate nouns \citep{Wolfart1973,Wolfart1996}, with Animate nouns often not being locativized. 

Nêhiyawêwin is a head-marking language, and so the person and number of the possessor is marked on the possessum. Singular possessors are marked only with prefixes: \{ni-\} for first person, \{ki-\} for second person, and \{o-\} for third person. For plural possessors, circumfixes are used: the prefixes are the same as for singular persons, which are matched with a set of suffixes: \{ni- -(i)n\^{a}n\} for first person plural exclusive (`ours but not yours'), \{ki- -(i)naw\} for first person plural inclusive (`mine/ours and yours'), \{ki- -(i)w\^{a}w\} for second person plural (`yours but not ours'), and \{o- -(i)w\^{a}w\} for third person plural. Nêhiyawêwin also distinguishes between alienable and inalienable nouns; the latter category must occur with possession and includes kinship terms and body parts as well as some other intimate possessions or relationships, such as \textit{nôhkom} `my grandmother' versus \textit{*ôhkom} `grandmother' \citep{Wolfart1973,Wolfart1996,Wolvengrey2011}. Some nouns, particularly body parts, are inalienable and may be posessed by a general posessor, as in \textit{mitas}, `(someone's) pants'.

Within Animate nouns, a pragmatic distinction is made regarding the topicality of a noun when used in the third person. All animate nouns can occur as either proximate third person (more topical entity in a discourse) and the obviative third person (less topical entity or entities in the discourse). This distinction occurs any time more than one animate third person occurs in a discourse, such as when one third person animate entity acts on another or when a third person animate entity possesses another, as in (\ref{obvi}). An obviative animate noun is marked with the obviative suffix \{-a\} and no number distinction is made; this is conventionally marked with 3^{\prime} (or as the `4th person', with no number distinction). The further obviative, which occurs when two obviative entities occur in one discourse, necessitating the demotion of one of them, is by convention marked with 3^{\prime}^{\prime} (or as the `5th person', also with no number distinction). As obviation is based in topicality rather than syntactic roles, it is generally not considered a marker of case. This is further exemplified with respect to verbal constructions below.

\begin{exe}
\ex
\gll atim nâpêw-a tahkwam-ê-w \\
dog.\textsc{prox} man-\textsc{obv} bite.\textsc{ta}-\textsc{dir.thm}-3\textsc{sg}.3^{\prime} \\
\trans `the (proximate) dog bites the (obviative) man.'
\label{obvi}
\end{exe}


\section{Verbs}

Cree verbs are traditionally classified according to both their transitivity and the animacy of their arguments/participants. There are two classes of intransitive verbs, which can occur with one inanimate participant (VII---verb inanimate intransitive) or one animate participant (VAI---verb animate intransitive). The former includes impersonal verbs such as weather terms and stative verbs used attributively to describe inanimate objects, and the latter includes intransitive actions and attributive verbs used to describe animate objects \citep{Bloomfield1946,Wolfart1973,Wolfart1996,okimasis2018cree}. The VII and VAI classes are exemplified in (\ref{vii1}) and (\ref{vai1}) respectively.

\begin{exe}
\ex VII
\begin{xlist}
\ex
\gll w\^{a}pisk\^{a}-w \\
be.white.\textsc{vii}-3\textsc{sg} \\
\trans `it is white'
\ex
\gll astotin w\^{a}pisk\^{a}-w \\
hat.\textsc{ni} be.white.\textsc{vii}-3\textsc{sg} \\
\trans `the hat (inanimate) is white'
\end{xlist}
\label{vii1}
\end{exe}

\begin{exe}
\ex VAI
\begin{xlist}
\ex
\gll wâpiskisi-w \\
be.white.\textsc{vai}-3\textsc{sg} \\
\trans `s/he (animate) is white' \\
\ex
\gll m\^{i}ciso-w \\
eat.\textsc{vai}-3\textsc{sg} \\
\trans `s/he eats, has a meal' \\
\end{xlist}
\label{vai1}
\end{exe}

Similarly, there are two classes of transitive verbs, though these are distinguished by the animacy of their second participant, often considered the object: transitive inanimate verbs (VTI) with an animate subject and an inanimate object, and transitive animate verbs (VTA) with two animate arguments.\footnote{\textit{Subjects} and \textit{objects} are conventionally called \textit{actors} and \textit{goals} in Algonquian literature \citep{Bloomfield1946, Wolvengrey2011}. \textit{Actors} here refer to the do-er of an action or subject of a description, despite the syntactic or semantic role. Similarly, \textit{goals} are any entity that receives a transitive action, regardless of the semantic or syntactic role (e.g. patient, recipient, benefactive, etc.). For this dissertation, I make use of these terms.} Examples are given in (\ref{vti1}) and (\ref{vta1}); note that there are three different verbs for `eat' depending on the transitivity and the animacy of participants.
\begin{exe}
\ex VTI
\begin{xlist}
\ex
\gll m\^{i}ci-w \\
eat.\textsc{vti}-3\textsc{sg} \\
\trans `s/he eats it (inanimate)' 
\end{xlist}
\label{vti1}
\end{exe}

\begin{exe}
\ex VTA 
\begin{xlist}
\ex
\gll mow-\^{e}-w \\
eat.\textsc{vta}-\textsc{dir.thm}-3\textsc{sg}.\textsc{actor}.3^{\prime}\textsc{goal} \\
\trans `s/he eats it/him (animate)' \\
\end{xlist}
\label{vta1}
\end{exe}




As noted above, Nêhiyawêwin does not have a case system to determine syntactic roles. Nouns exhibit obviation, a system in which non-focal, animate, third persons are marked \citep[94]{Bloomfield1946}. Together with the directionality system, discussed below, semantic roles are determined through relationships between items rather than simple case marking.


Verbs agree with arguments according to animacy: inanimate actors for VII and animate actors for VAI, VTI, and VTA. The inanimate participant in a clause containing a VTI is the goal of the verb, or some other oblique argument, but not the actor. The person marking on VII, VAI, and VTI verbs corresponds to the person and number of the actor. However, in VTAs, both arguments are animate and realized in the verbal morphology, with their respective roles determined by obviation and direction morphology, discussed below. Essentially, verbs and their arguments can be thought of as constructions where certain verb stems license a certain number of arguments of particular animacy.



To determine the roles of participants in VTA clauses, Algonquian languages make use of a direct-inverse system \citep{Wolfart1973,jacques2014direct}. VTAs occur with two animate participants and there is no grammatical case or fixed word order by which to determine the semantic roles. Instead, direction is used as a method of determining which argument is the actor and which is the goal. In Nêhiyawêwin, direction is determined by the relative topicality of participants, extended beyond the proximate-obviative distinction into a full hierarchy known as the Algonquian person hierarchy, given in (\ref{hier1}) \citep{jolley1983algonquian}. Direction is indicated by a theme morpheme, which indicates that the action is either \textit{direct} or \textit{inverse}. When a more topical participant acts on a less topical participant, the morphology or theme sign is direct (-\textit{\^{a}}-, -\textit{\^{e}}-, -\textit{i}-). When the opposite occurs, the morphology or theme sign is inverse (-\textit{ik(w/o)}-, -\textit{iti}-). As visualized in (\ref{hier1}), second person is ranked topically above first person, and both of these speech act participants are ranked above all third or unspecified\footnote{In Nêhiyawêwin, the Unspecified Actor is an actor on a verb where the exact person and number of the actor is not specified. It may be translated as a sort of agentless passive \citep{Wolvengrey2011}.} persons, wherein obviation applies. Due to this hierarchy, first person acting on second necessarily always occurs with inverse morphology. In this way, these are not passive forms, but simply the only way of indicating first person acting on second. For this and a variety of other reasons not discussed herein, Nêhiyawêwin inverse forms are not considered equivalent to passive voice in languages such as English \citep{Wolfart1973,Wolvengrey2011}.



\begin{exe} % sets up the top-level example environment
\ex\label{hier1} 2 $>$ 1 $>$ Unspecified Actor $>$ 3 $>$ 3^{\prime} $>$ 3^{\prime}^{\prime}
\end{exe}


With obviation marked on both nouns and verbs, sentences such as those in (\ref{vtasentence1})a. are possible in Nêhiyawêwin. Additionally, both obviative and further obviative marking may be needed, depending on the number of third persons lexically specified, as in (\ref{vtasentence1})b. However, when a Nêhiyawêwin VTI is involved, and so there is an inanimate goal rather than an animate one, no goal or obviative marking occurs on either the verb, or the inanimate noun, as in (\ref{vtisentence1}) \citep{Wolfart1973,Wolvengrey2011}. \\

\begin{exe}
\ex VTA
\begin{xlist}
\ex
\gll c\^{a}n pahkw\^{e}sikan-a { } mow-\^{e}-w \\
John.3\textsc{sg} bread.\textsc{na}-3^{\prime} { } eat.\textsc{vta}-\textsc{thm.dir}-3\textsc{sg}.3^{\prime} \\
\trans `John eats bread (animate).'
\ex
\gll c\^{a}n o-t\^{e}m-a osk\^{a}t\^{a}skw-a { } mow-\^{e}-yiwa \\
John.3\textsc{sg} 3.\textsc{poss}-dog.\textsc{na}-3^{\prime} carrot.\textsc{na}-3^{\prime}^{\prime} { } eat.\textsc{vta}-\textsc{thm.dir}-3^{\prime}.3^{\prime}^{\prime} \\
\trans `John's (3\textsc{sg}) dog (3^{\prime}) eats the carrot (animate, 3^{\prime}^{\prime}).'\footnote{As the marking for obviative and further obviative is formally the same, they must instead be distinguished on the basis of semantics and pragmatics.}
\end{xlist}
\label{vtasentence1}
\end{exe}

\begin{exe}
\ex VTI
\begin{xlist}
\ex
\gll c\^{a}n wiy\^{a}s m\^{i}ci-w \\
John.3\textsc{sg} meat.\textsc{ni} eat.\textsc{vti}-3\textsc{sg}\\
\trans `John eats meat (inanimate).'
\end{xlist}
\label{vtisentence1}
\end{exe}

The \{-w\} in (\ref{vtisentence1}) is one of two third person suffixes in the VTIs, the other being \{-m\}. This morph is homophonous with third person markers in other conjugation classes. Alongside extensive person and direction morphology, several other categories may also be expressed on verbs.\footnote{For a large (though not yet complete) overview of Nêhiyawêwin morphemes (including common preverbs) see \citet{CookMuehl2010}.} Preverbs attach to the verb between person and the verb stem and serve several purposes. There are two types of preverbs: grammatical and lexical. The outermost of grammatical preverbs include those such as \{\^{e-}\} and other Conjunct preverbs including \{ka-\}/\{ta-\}\footnote{This is a single morpheme that contains two allomorphs that are used in free variation. In central and southern Alberta, \{ka-\} is the more common form}, and \{kâ-\}. While most preverbs are relatively freely combineable, these three are mutually exclusive. These morphs serve as complementizers and may have further functions, such as marking future or relative clauses. Closer to the verbal stem, one can observe another type of grammatical preverb for tense and aspect: \{k\^{i}-\} for past, \{w\^{i}-\} for prospective future, and \{ka-/ta\}- for definite future. Closer still to the verb are lexical preverbs, e.g., \{kakw\^{e}-\} `try (to)', \{niht\^{a}-\} `be good at', \{nitawi-\} `go and (do something)', \{\^{a}piht\^{a}-\} `half (of)/halfway', \{kihci-\} `large', etc. \citep{Wolfart1973,Wolfart1996,Wolvengrey2001}, though even these show a gradience in lexicality/grammaticality.




\section{Nêhiyawêwin Order}
\label{SecOrder}
 Algonquian languages are noted for their unique system of what is called Order, most easily recognized through allomorphy instantiated on the person-marking affixes of verbs. According to \citet[97]{Bloomfield1946}:

\begin{quote}
The forms of the verb fall into five orders. Each order consists of one or more modes, each with a full set of forms. The independent order takes prefixes; its principal mode, however, the indicative, has zero instead of \textit{we-} for the third person. The other orders take no prefixes. The imperative has forms for second person actor only, and only one mode. The prohibitive has two modes with the same restriction, but also a third mode, the potential, with a full set of forms. The conjunct and interrogative orders are used only in subordinate clauses and as participles. The languages differ widely in their stock of modal forms; all seem to have lost a few, and some languages have created new ones.
\end{quote}

The Orders described by Bloomfield are mutually exclusive. One can not have the morphology for both the Independent and the Conjunct, for example. For this reason, it seems obvious to group the  mutually exclusive orders as a cohesive unit. As alluded to by Bloomfield, some Algonquian languages have fewer than five orders. Nêhiyawêwin is one of these languages, usually regarded as having only three orders: the Imperative, the Independent and the Conjunct. Despite the centrality of Order to the use of verbs, descriptions of the system as a whole remain vague for Nêhiyawêwin. Sometimes, Order is treated as a semantic alternation: the Imperative Order marks the imperative mood, while the  Independent and Conjunct do not correspond to any specific mood. However, there is no such distinction between the Independent and Conjunct Orders. Instead, these Orders are usually analyzed through their morphological difference.

I argue that Order, can be analyzed as an alternation. I suggest that Order as currently described is essentially two overlapping linguistic systems: one of mood/aspect and one of morphology that corresponds to an type of semantic alternation previously undescribed: a paradigmatic alternation. To support this proposal, I will detail the morphological, syntactic, and semantic/pragmatic ways in which Order is used and defined, and the ways in which these definitions are inadequate.

\subsection{Morphology}
Speaking strictly in terms of structural/morphological phenomena, the different Orders of Nêhiyawêwin can be divided into three main types: the Independent, the Conjunct, and the Imperative. The Independent is comprised of those forms which mark for any person argument and take a person prefix (\{ni-\} for first person, \{ki-\} for second, and no prefix for third person) and a set of suffixes \citep{Bloomfield1946,Wolfart1973}. The Conjunct is comprised of forms that also mark for any argument and which take no person prefix and one of a number of conjunct suffixes. The Imperative, on the other hand, marks for only second person arguments, cannot be used without a second person argument, does not make use of person prefixes, and uses a unique set of suffixes as compared to the Independent or Conjunct. Treating these Orders as of the same type due to their mutual exclusivity, as done by \citet{Bloomfield1946}, results in a system depicted in Figure \ref{fig:triorder}. 

\begin{figure}[h]
\centering
\Tree[.Order [.Independent ] [.Conjunct ] [.Imperative ]]
\caption{Order Ontology based on Morphology}
\label{fig:triorder}
\end{figure}

This ontology, however, fails to capture clear distinction of the Imperative from the Independent and the Conjunct. In a strictly structural sense, the shape of the Independent and Conjunct paradigms are similar to each other, while the Imperative's diverges from this standard substantially. To demonstrate and describe these differences, the structural makeup of the three canonical orders will be described below.

\subsubsection{The Independent Order}

According to Wolfart, the Independent order comes in two main modes: the preterit and non-preterit \citeyearpar{Wolfart1973}. Preterit forms can be thought of as past-perfect constructions; conversely, the non-preterit form is essentially equivalent to the traditionally described indicative \citep{Wolfart1973}. Wolfart spends much of his description discussing the preterit forms of the Independent order, explaining the three types of preterit Independents. Since Wolfart's publication these preterit forms have largely fallen out of use in Nêhiyawêwin \citep[74]{Wolvengrey2011} and so will not be further discussed. As previously mentioned, the Independent is identified by \citet{Bloomfield1946}, \citet{Wolfart1973}, and \citet{Cook2014} as the Order that marks for all possible persons with the person prefixes \{ni-\} and \{ki-\} for first and second persons, respectively, and the lack of an prefix for the third and obviative persons. Independent forms are unable to take the \{ê-\} preverb (discussed later) which has begun to function exclusively as a marker of Conjunct constructions.

\begin{table}
  \centering
  \begin{tabular}{lllllllll}
    \toprule
     & Prefix & Stem     & Theme & SAP Person & Obviative & 3\textsc{sg} & 3\textsc{pl} & 3' \\
    \midrule
3\textsc{sg}  &        & mihkwâ &       &            &           & w           &            &    \\
3\textsc{pl}  &        & mihkwâ &       &            &           & w          & a          &    \\
3'\textsc{sg} &        & mihkwâ &       &            & yi        & w          &            &    \\
3'\textsc{pl} &        & mihkwâ &       &            & yi        & w          & a          &   \\
    \bottomrule
  \end{tabular}
  \caption{
    VII Independent Paradigm. Based on \citet[413]{Wolvengrey2011}. \label{tab:viiindpara}
  }
\end{table}

Table \ref{tab:viiindpara} describes the Independent VII paradigm. Notice that only third person (and obviative) participants exist in this paradigm, and so no speech act participant prefix or suffixes are used. These along with the final column, the additional third person obviative suffix, are unused but included to maintain consistency with the VAI, VTI, and VTA paradigms. 


The VTA paradigms are further split. Here, a distinction is made between the \textit{local} and \textit{mixed} subsets. A \textit{local} VTA subparadigm is one where the actor and the goal are both speech act participants (first or second persons), while the \textit{mixed} subparadigm contains interactions between speech act participants and third or obviative persons. This subparadigm also contains third persons acting on obviative persons. This is presented in this way for the sake of convenience. In reality, one could place these non-speech act participant forms in their own sub-paradigm.

\begin{table}
    \begin{tabular}{lllllllll}
    \toprule
     & Prefix & Stem  & Theme & SAP Person & Obviative & 3\textsc{sg} & 3\textsc{pl} & 3' \\
    \midrule
    
    1    & ni     & wâpam & â     &            &           & w          & ak         &    \\
2\textsc{sg}    & ki     & wâpam & â     &            &           & w          & ak         &    \\
1\textsc{pl}  & ni     & wâpam & â     & nân        &           &            & ak         &    \\
21\textsc{pl} & ki     & wâpam & â     & naw        &           &            & ak         &    \\
2\textsc{pl}  & ki     & wâpam & â     & wâw        &           &            & ak         &    \\
3\textsc{sg}    &        & wâpam & ê     &            &           & w          &            &    \\
3\textsc{pl}  &        & wâpam & ê     &            &           & w          & ak         &    \\
3'   &        & wâpam & ê     &            & yi        & w          &            & a \\
    \bottomrule
  \end{tabular}
  \caption{
    VTA Independent Direct, Mixed Participant Paradigm Excerpt. Based on \citet[418]{Wolvengrey2011}.\label{tab:vtadir}
  }
\end{table}

As seen in Tables \ref{tab:vaiindpara} and \ref{tab:vtiindpara}, the paradigms of the VAI and VTI are extremely similar, differing in their inclusion of a theme sign.\footnote{Theme is used in the sense of traditional grammars, such as \citet{goodwin}, where the theme sign is used to associate a stem with a particular paradigmatic shape.} 


\begin{table}[h]
  \centering
\begin{tabular}{lllllllll}
\toprule
     & Prefix & Stem & Theme & SAP Person & Obviative & 3\textsc{sg} & 3\textsc{pl} & 3' \\
    \midrule

1\textsc{sg}    & ni     & nipâ &       & n          &           &            &            &    \\
2\textsc{sg}    & ki     & nipâ &       & n          &           &            &            &    \\
1\textsc{pl}  & ni     & nipâ &       & nân        &           &            &            &    \\
21\textsc{pl}\tablefootnote{This represents the first person inclusive actor. In Algonquian linguistics, this is often considered as a second person form due to its morphology and its marking with the second person \{ki-\} prefix in the Independent.} & ki     & nipâ &       & (nâ)naw    &           &            &            &    \\
2\textsc{pl}  & ki     & nipâ &       & nâwâw      &           &            &            &    \\
3\textsc{sg}    &        & nipâ &       &            &           & w          &            &    \\
3\textsc{pl}  &        & nipâ &       &            &           & w          & ak         &    \\
3'   &        & nipâ &       &            & yi        & w          &            & a \\

    \bottomrule
  \end{tabular}
  \caption{
    VAI Independent Paradigm. Based on \citet[415]{Wolvengrey2011}.\label{tab:vaiindpara}
  }
\end{table}

\begin{table}[h]
  \centering
\begin{tabular}{lllllllll}
    \toprule
     & Prefix & Stem   & Theme & SAP Person & Obviative & 3\textsc{sg} & 3\textsc{pl} & 3' \\
    \midrule
1\textsc{sg}    & ni     & wâpaht & ê     & n          &           &            &            &    \\
2\textsc{sg}    & ki     & wâpaht & ê     & n          &           &            &            &    \\
1\textsc{pl}  & ni     & wâpaht & ê     & nân        &           &            &            &    \\
21\textsc{pl} & ki     & wâpaht & ê     & naw        &           &            &            &    \\
2\textsc{pl}  & ki     & wâpaht & ê     & wâw        &           &            &            &    \\
3\textsc{sg}    &        & wâpaht & am    &            &           & (w)        &            &    \\
3\textsc{pl}  &        & wâpaht & am    &            &           & w          & ak         &    \\
3'   &        & wâpaht & am    &            & (i)yi       & w          &            & a    \\
    \bottomrule
  \end{tabular}
  \caption{
    VTI Independent Paradigm. Based on \citet[417]{Wolvengrey2011}. Note the difference of theme sign for local and non-local participants. \label{tab:vtiindpara}
  }
\end{table}



In fact, there are some VAIs, like \textit{âsokâham}, `s/he swims across' that follow the general VTI paradigm and takes the \{-m-\} theme sign; conversely, some VTIs like \textit{kâtâw}, `S/he hides something,' take VAI morphology and follow the VAI paradigm. This has lead to an alternative interpretation of verb conjugation proposed by \citet{Wolvengrey2011}. Here, there is a three-way distinction between verbs, based solely on the number of animate participants: \textit{V0} containing any verb forms with no animate participants (corresponding to VII), \textit{V1} containing verbs with only one animate participant (corresponding to VAI and VTI), and \textit{V2} containing verbs with two animate participants (corresponding to VTA).

\begin{table*}[h]
\centering
\begin{tabular}{llllllllll}
    \toprule
        & Prefix & Stem  & Theme & 1\textsc{sg} & 1\textsc{pl} & 2\textsc{pl} &  &  &  \\
    \midrule

2\textsc{sg}     & ki     & wâpam & i     & n  &     &       &  &  &  \\
2\textsc{sg/pl} & ki     & wâpam & i     & n & ân     &       &  &  &  \\
2\textsc{pl}     & ki     & wâpam & i     &    &     & nâwâw &  &  &  \\    
\bottomrule
  \end{tabular}
  \caption{
    VTA Independent Direct, Local Paradigm Excerpt. Adapted from \citet[418]{Wolvengrey2011}.\label{tab:vtadirlocal}
  }
\end{table*}

\begin{table*}
\resizebox{\textwidth}{!}{%
\begin{tabular}{lllllllll}
    \toprule
     & Prefix & Stem   & Theme & SAP Person & Obviative & 3rd Person & 3\textsc{pl} & 3' \\
    \midrule
1    &    ni    & wâpam & ik(w) &            &           & w          & ak         &    \\
2    &  ki      & wâpam & ik(w) &            &           & w          & ak         &    \\
1pl  &    ni    & wâpam & ikw   & inân       &           &            & ak         &    \\
21pl &      ki  & wâpam & ikw   & inaw       &           &            & ak         &    \\
2pl  &  ki      & wâpam & ikw   & iwâw       &           &            & ak         &    \\
3    &        & wâpam & ik(w) &            &           & w          &            &    \\
3pl  &        & wâpam & ik(w) &            &           & w          & ak         &    \\
3'   &        & wâpam & ikw   &            & yi        & w          &            & a  \\
    \bottomrule
  \end{tabular}}
  \caption{
    VTA Independent Inverse, Mixed Participant Paradigm Excerpt. \citet[418]{Wolvengrey2011}. \label{tab:vtainv}
  }
\end{table*}

Tables \ref{tab:vtadir} through \ref{tab:vtainvlocal} gives a subset of an Independent VTA paradigm,\footnote{There are 36 person combinations in each of the Independent and Conjunct Orders.} exemplifying direct and inverse forms for different pairs of participants for the VTA \textit{w\^{a}pam\^{e}w} `s/he (animate) sees someone (animate)'. The person prefixes, and often the suffixes, remain the same while the direction morphology changes (note that some dialects allow for third person inverse forms with \{-ikow\} endings instead of \{-ik\}\footnote{Note that the\{iko\} morph derives from the \{ikw\} morpheme along with a an epenthetic /i/, the combination of which produces /iko/}.). While the VTA Independent forms are decomposible, the Conjunct forms are not always so predictable.

\begin{table}[h]
\centering
\begin{tabular}{llllll}
    \toprule
        & Prefix & Stem  & Theme & 1\textsc{sg/pl} & 2\textsc{pl}  \\
    \midrule
2\textsc{sg}     & ki     & wâpam & iti   & n       &         \\
2\textsc{sg/pl} & ki     & wâpam & iti   & nân     &        \\
2\textsc{pl}     & ki     & wâpam & iti   &         & nâwâw  \\
    \bottomrule
  \end{tabular}
  \caption{
    VTA Independent Inverse, Local Paradigm Excerpt. Based on \citet[418]{Wolvengrey2011}. \label{tab:vtainvlocal}
  }
\end{table}

\FloatBarrier
\subsubsection{The Conjunct Order}

\citet{Wolfart1973} described four modes of the Conjunct, based on the presence or absence of the verb-final suffix \{-ih\} and the presence or absence of `initial change' (an Algonquian process where the first vowel in the verb stem (or sometimes verbal prefixes) is mutated---abbreviated IC) \citep{Wolfart1973}. According to Wolfart those Conjunct verbs with both \{-ih\} and Initial Change are iterative and are named by him as such. Those without Initial Change but with \{-ih\} impart conditionality and are what Wolfart terms the \textit{subjunctive}. Verbs with Initial Change but without \{-ih\} are simply called \textit{Changed} and are the most commonly used Conjunct form, though Wolfart notes that Initial Change is beginning to fall out of use, being replaced instead by the use of the \{\^{e}-\} preverb (\citeyear{Wolfart1973}). This view is consistent with that of Wolvengrey's account of \{\^{e}-\} being born out of a regularization of a particular type of change, $/$i$/$ $>$ $/$\^{e}$/$, where the changed vowel was extracted from the construction to be used as a preverb, the verb stem retaining its original form (e.g., \textit{itw\^{e}t} $>$ \textit{\^{e}tw\^{e}t} $>$ \textit{\^{e}-itw\^{e}t})  (A. Wolvengrey, Personal Communication).  Finally, those verbs without Initial Change or \{-ih\} are referred to as \textit{simple} \citep{Wolfart1973}. A summary of this four way distinction is found in Table \ref{Conjunctforms}. In more contemporary Nêhiyawêwin orthography, the {-ih} ending is realized simply as a suffixal \{-i\}. 

\begin{table}
\centering
\begin{tabular}{llll} \\
\toprule
     &   & Initial Change                  &                  \\
\midrule
     &   & +                        & -                      \\
     & + & Iterative (`whenever it is') & Subjunctive (`if it be') \\
/ih/ &   &                          &                        \\
     & - & Changed (`it being')       & Simple (`that it is') \\   \bottomrule
\end{tabular}
\caption{Wolfart's Conjunct modes. Adapted from \citet[45]{Wolfart1973}}
\label{Conjunctforms}
\end{table}

Cook (\citeyear{Cook2014}) provides further detail on the morphosyntactic and semantic behaviour of the Conjunct order. Agreeing with Wolfart (\citeyear{Wolfart1973}), Cook explains the wide spread use of the order through several modes of the Conjunct. Unlike Wolfart's tetrachotamy, Cook gives a pentachotomy (\citeyear{Cook2014}). Under Cook's system, the Conjunct is split into the \textit{Changed} and \textit{Unchanged} modes (\citeyear{Cook2014}). The Changed Conjunct is further split into three subtypes: the \textit{Changed Conjunct}$_{1}$, the \textit{Changed Conjunct}$_{2}$, and the \textit{Iterative Changed Conjunct}\footnote{Where \citet{Wolfart1973} identified an iterative/conditional morpheme as {-ih}, \citet{Cook2014} follows the contemporary orthography.}. Although three subtypes are titled \textit{Changed} due to being historically derived from changed forms, only the Iterative currently exhibits Initial Change. Changed$_{1}$ and Changed$_{2}$ on the other hand, are marked with the \{\^{e}-\} and \{k\^{a}-\} preverbs respectively \footnote{\citet{Wolfart1973} classifies these two types together as changed conjunct forms, deriving \{k\^{a}-\} from \{k\^{i}-\})}. The unchanged Conjunct forms are split into the \textit{Subjunctive Simple Conjunct}, which are marked with no preverb and no Initial Change (but instead with a {-i} suffix appended to the person endings), and the \textit{Irrealis Simple Conjunct}, which is marked with the \{ka-\} preverb. These forms are represented in Table \ref{CookConjone}.\footnote{Terminology for these terms vary between researchers. The subjunctive is sometimes referred to as the \textit{future conditional}, which helps avoid the the term subjunctive \citep{Ratt2016,okimasis2018cree}. Similarly, the term \textit{timeless conditional} has been used in place of \textit{iterative} \citep{HarriganVAIPara}. }

\begin{table}
\centering
\begin{tabular}{llll} \\
\toprule
Submode   & Subtype     & Form    & Gloss    \\
\midrule
Changed   & Changed Conjunct$_{1}$    & \^{e}-apiy\^{a}n & `I sleep' \\
          & Changed Conjunct$_{2}$   & k\^{a}-apiy\^{a}n & `When I sleep' \\
          & Iterative   & \^{e}piy\^{a}ni  &  `Whenever I sleep' \\
Unchanged & Simple      & ka-apiy\^{a}n &  `for him to eat'\\
          & Subjunctive & apiy\^{a}ni  &  `whenever I eat'\\
\bottomrule
\end{tabular}
\caption{Cook's Conjunct modes. Adapted from \citet[125]{Cook2014}}
\label{CookConjone}
\end{table}
\FloatBarrier

The following paradigms demonstrate the general shape of the Conjunct paradigm and represent the ê-Conjunct forms for the VII, VAI, VTI, and VTA conjunct classes. 

As with the Independent paradigm, the VII Conjunct paradigm marks only for the third and obviative persons, as in Table \ref{tab:viicnjpara}. 
\begin{table}[h]
  \centering
\begin{tabular}{lllllllll}
    \toprule
     & Prefix & Stem     & Theme & SAP Person & Obviative & 3rd Person & 3\textsc{pl} & 3' \\
    \midrule
3\textsc{sg}  & ê-     & mihkwâ &       &            &           & k          &            &    \\
3\textsc{pl}  &  ê-     & mihkwâ &       &            &           & k          & i          &    \\
3'\textsc{sg} &  ê-      & mihkwâ &       &            & yi       & k          &            &    \\
3'\textsc{pl} &  ê-   & mihkwâ &       &            &yi       & k          & i          &     \\
    \bottomrule
  \end{tabular}
  \caption{
    VII Conjunct Paradigm for \textit{mihkwâ}, `to be red'. Based on \citep[413]{Wolvengrey2011} \label{tab:viicnjpara}
  }
\end{table}

\begin{table}[h]
  \centering
  \footnotesize
\begin{tabular}{lllllllll}
    \toprule
     & Prefix & Stem & Theme & SAP Person & Obviative & 3rd Person & 3\textsc{pl} & 3' \\
    \midrule
1\textsc{sg}    & ê      & nipâ &       & yân        &           &            &            &    \\
2\textsc{sg}    & ê      & nipâ &       & yan        &           &            &            &    \\
1\textsc{pl}  & ê      & nipâ &       & yâhk       &           &            &            &    \\
21\textsc{pl} & ê      & nipâ &       & yahk       &           &            &            &    \\
2\textsc{pl}  & ê      & nipâ &       & yêk        &           &            &            &    \\
3\textsc{sg}    & ê      & nipâ &       &            &           & t          &            &    \\
3\textsc{pl}  & ê      & nipâ &       &            &           & c          & ik         &    \\
3'   & ê      & nipâ &       &            & yi        & t          &            &   \\
    \bottomrule
  \end{tabular}
  \caption{
    VAI Conjunct Paradigm for \textit{nipâ}, `to sleep'. Based on \citep[415]{Wolvengrey2011}.\label{tab:vaicnjpara}
  }
\end{table}

\begin{table}
  \centering
  \footnotesize
\begin{tabular}{lllllllll}
    \toprule
     & Prefix & Stem   & Theme & SAP Person & Obviative & 3rd Person & 3\textsc{pl} & 3' \\
    \midrule
1\textsc{sg}    & ê      & wâpaht & am    & ân         &           &            &            &    \\
2\textsc{sg}    & ê      & wâpaht & am    & an         &           &            &            &    \\
1\textsc{pl}  & ê      & wâpaht & am    & âhk        &           &            &            &    \\
21\textsc{pl} & ê      & wâpaht & am    & ahk        &           &            &            &    \\
2\textsc{pl}  & ê      & wâpaht & am    & êk         &           &            &            &    \\
3\textsc{sg}    & ê      & wâpaht & am    &            &           & k          &            &    \\
3\textsc{pl}  & ê      & wâpaht & am    &            &           & k          & ik         &    \\
3'   & ê      & wâpaht & am    &            & (i)yi       & t          &            &       \\
    \bottomrule
  \end{tabular}
  \caption{
    VTI Independent Paradigm for \textit{wâpaht}, `to see it'. Based on \citep[417]{Wolvengrey2011}. \label{tab:vticnjpara}
  }
\end{table}

Similar to the Independent, the Conjunct's VAI and VTI paradigms are strikingly similar. The main difference is the inclusion of an epenthetic /y/ in the SAP Person endings for the VAI paradigm, as well as the /am/ theme element in the VTI. These differences are exemplified in the differences between Tables \ref{tab:vaicnjpara} and \ref{tab:vticnjpara}.







\begin{table}[h]
  \centering
\begin{tabular}{llllll}
    \toprule
Actor $\rightarrow$   Goal                                                                               & Prefix & Verb Stem & Theme & 2\textsc{sg}/2\textsc{pl} & 1\textsc{pl} \\    
\midrule
2\textsc{sg} $\rightarrow$   1\textsc{sg}                              & ê-     & mow          & i     & yan                                                         &                               \\
2\textsc{sg}/\textsc{pl} $\rightarrow$   1\textsc{pl}                  & ê-     & mow          & i     &                     & yâhk                          \\
2\textsc{pl} $\rightarrow$   1\textsc{sg}                              & ê-     & mow          & i     & yêk                                                         &                               \\
    \bottomrule
  \end{tabular}
  \caption{
    VTA Conjunct Direct, Local Paradigm Excerpt for \textit{mow}, `to eat'. Based on \citet[419]{Wolvengrey2011}. \label{tab:vtacnjdir}
  }
\end{table}


\begin{table}[h]
\resizebox{\textwidth}{!}{%
\begin{tabular}{llllllll}
    \toprule
Actor $\rightarrow$   Goal                                                                                                                           & Prefix & Verb Stem & Theme & Obviative & SAP   & 3\textsc{sg} & 3\textsc{pl} \\
    \midrule
1\textsc{sg} $\rightarrow$   3\textsc{sg}        & ê-     &    mow       &       &           & it    &                               & ik                            \\
2\textsc{sg} $\rightarrow$   3\textsc{sg}        & ê-     & mow          &       &           & isk    &                               & ik                            \\
3\textsc{sg} $\rightarrow$   3'                  & ê-     &    mow       & iko     &           & yâhk  &                               & ik                            \\
1\textsc{pl} $\rightarrow$   3\textsc{sg}        & ê-     &       mow    & iko     &           & yahkw &                               & ik                            \\
21\textsc{pl} $\rightarrow$   3\textsc{sg}       & ê-     &          mow & iko     &           & yêkw  &                               & ik                            \\
2\textsc{pl} $\rightarrow$ 3\textsc{sg}          & ê-     &     mow      & iko     &           &       & t                             &                               \\
3\textsc{pl} $\rightarrow$   3'                  & ê-     &     mow      & iko     &           &       & t                             & ik                            \\
3' $\rightarrow$   3''                           & ê-     &     mow      & iko     & yi        &       & t                             &                              \\
    \bottomrule
  \end{tabular}}
  \caption{
    VTA Conjunct Inverse, Mixed Participant Paradigm Excerpt. Based on \citet[419]{Wolvengrey2011}. \label{tab:vtacnjinvmixed}
  }
\end{table}


\begin{table}
\centering
\begin{tabular}{llllll}
    \toprule
Actor $\rightarrow$   Goal                                                                               & Prefix & Verb Stem & Theme & 2\textsc{sg}/2\textsc{pl} & 1\textsc{pl} \\    
\midrule
2\textsc{sg} $\rightarrow$   1\textsc{sg}             & ê-     &    mow       & i     & yan                                                         &                               \\
2\textsc{sg}/\textsc{pl} $\rightarrow$   1\textsc{pl} & ê-     &    mow       & i     &                                                             & yâhk                          \\
2\textsc{pl} $\rightarrow$   1\textsc{sg}             & ê-     &    mow       & i     & yêk                                                         &                               \\
    \bottomrule
  \end{tabular}
  \caption{
    VTA Conjunct Inverse, Local Paradigm Excerpt. Based on \citet[419]{Wolvengrey2011}. \label{tab:vtacnjinvlocal}
  }
\end{table}


\begin{table}
\begin{tabular}{llllllll}
    \toprule
Actor $\rightarrow$   Goal                                                                                                                           & Prefix & Verb Stem & Theme & Obviative & SAP   & 3\textsc{sg} & 3\textsc{pl} \\
    \midrule
1\textsc{sg} $\rightarrow$   3\textsc{sg}       & ê-     &  mow         &       &           & ak    &                               & ik                            \\
2\textsc{sg} $\rightarrow$   3\textsc{sg}       & ê-     &  mow         &       &           & at    &                               & ik                            \\
3\textsc{sg} $\rightarrow$   3'                 & ê-     &  mow         & â     &           & yâhk  &                               & ik                            \\
1\textsc{pl} $\rightarrow$   3\textsc{sg}       & ê-     &  mow         & â     &           & yahkw &                               & ik                            \\
21\textsc{pl} $\rightarrow$   3\textsc{sg}      & ê-     &  mow         & â     &           & yêkw  &                               & ik                            \\
2\textsc{pl} $\rightarrow$ 3\textsc{sg}         & ê-     &  mow         & â     &           &       & t                             &                               \\
3\textsc{pl} $\rightarrow$   3'                 & ê-     &  mow         & â     &           &       & t                             & ik                            \\
3' $\rightarrow$   3''                          & ê-     &  mow         & â     & yi        &       & t                             &                              \\
    \bottomrule
  \end{tabular}
  \caption{
    VTA Conjunct Direct, Mixed Participant Paradigm Excerpt. Based on \citet[419]{Wolvengrey2011}. \label{tab:vtacnjdir2}
  }
\end{table}



The paradigmatic breakdowns used in Tables \ref{tab:vtacnjdir} through \ref{tab:vtacnjdir2} highlight the theme morphs for the direct and inverse. There are alternative ways to analyze the endings in VTA paradigms, perhaps more straightforwardly by chunking all the suffixes together as sorts of portmanteau morphemes, as in \citet{harrigan2017learning}. For consistency and compatibility with \citet{Wolvengrey2011}, this dissertation will continue to use the paradigmatic patterns as presented in the four-conjugation class appendices of \citet{Wolvengrey2011}.


\subsubsection{The Imperative Order}


Just as \citet{Bloomfield1946} does, \citet{Wolfart1973} describes two main Imperative modes: the Immediate and Delayed imperatives. The Immediate Imperative refers to a command to do something immediately, while the Delayed Imperative refers to a command to do something later. Because the Imperative only encodes command forms, both the immediate and the delayed mark only for second person forms. Resultingly, VII conjugation class of verbs, which only encodes third person and obviative actors, does not occur in the Imperative.

Across the remaining three conjugation classes, the Immediate Imperative describes an immediate command and is marked with no suffix, a \{-tân\} suffix, and a \{-k\} suffix for second person singular, first person inclusive, and second person plural, respectively.
Again, the main differentiation between the VAI and VTI imperative paradigms is the latter containing a theme morph, as seen in Tables \ref{tab:vaiimppara} and \ref{tab:vtiimppara}.

\begin{table}[h]
  \centering
  \begin{tabular}{llll}
    \toprule
              &       Verb Stem                         & Immediate & Delayed \\
    \midrule
2\textsc{sg}  &      nipâ                          &           &         \\
21\textsc{pl} &      nipâ                          & tân       &         \\
2\textsc{pl}  &     nipâ                           & k         &         \\
\hline
\hline
2\textsc{sg}  &     nipâ                           &           & hkan    \\
21\textsc{pl} &     nipâ                           &           & hkahk   \\
2\textsc{pl}  &     nipâ                           &           & hkêk    \\
    \bottomrule
  \end{tabular}
  \caption{
    VAI Imperative Paradigm. \citep[395]{Wolvengrey2011} \label{tab:vaiimppara}
  }
\end{table}

\begin{table}[h]
  \centering
  \begin{tabular}{lllll}
    \toprule
              & Verb Stem            & theme   & Immediate & Delayed \\
    \midrule
2\textsc{sg}  &   wâpaht          & a       &           &         \\
21\textsc{pl} &   wâpaht        & ê       & tân       &         \\
2\textsc{pl}  &   wâpaht            & amw     & ik         &         \\
\hline
\hline
2\textsc{sg}  & wâpaht            & amw     &           & ihkan    \\
21\textsc{pl} & wâpaht            & amw     &           & ihkahk   \\
2\textsc{pl}  & wâpaht            & amw     &           & ihkêk    \\
    \bottomrule
  \end{tabular}
  \caption{
    VTI Imperative Paradigm.  \citep[398]{Wolvengrey2011} \label{tab:vtiimppara}
  }
\end{table}

 Additionally, the second person plural and all delayed forms contain an empenthetic /\textipa{I}/. In each of these cases, the theme sign is realized as \{-amw-\} and the resulting /w\textipa{I}/ sequence coalesces to /o/, as in \textit{wâpahtamok}, `See it, y'all!'. Where the \{-amw-\} and epenthetic /\textipa{I}/ occur before an /h/, the surfacing form contains a long /o/, as in \textit{wâpahtamôhkan}, `see it later!'

\begin{table*}[h]
\centering
\begin{tabular}{llll|l||l|l}
    \toprule
                 &      &       & \multicolumn{2}{c}{Immediate}      & \multicolumn{2}{c}{Delayed}   \\
                 & Stem & Theme & 3\textsc{sg} & 3\textsc{pl} & 3\textsc{sg} & 3\textsc{pl}   \\ 
\midrule
2\textsc{sg}     & mow     &       & (i)           & ik          &              &                \\
21\textsc{pl}    & mow     & â      & tân           & ik          &              &                 \\
2\textsc{pl}     & mow     &       & ihkw          & ik          &              &                  \\
\hline
\hline
2\textsc{sg}     & mow    & â      &               &             & hkan          & ik                \\
21\textsc{pl}    & mow    & â      &               &             & hkahkw        & ik                 \\
2\textsc{pl}     & mow    & â      &               &             & hkêkw         & ik                  \\

    \bottomrule
  \end{tabular}
  \caption{
    VTA Imperative Mixed Participant Paradigm \citep[403]{Wolvengrey2011}. \label{tab:vtaimpmixed}
  }
\end{table*}


\begin{table*}[h]
\centering
\begin{tabular}{llll|l||l|l}
    \toprule
                 &      &       & \multicolumn{2}{c}{Immediate}      & \multicolumn{2}{c}{Delayed}   \\
                 & Stem & Theme & 1\textsc{sg} & 1\textsc{pl} & 1\textsc{sg}  & 1\textsc{pl}   \\ 
\midrule
2\textsc{sg}     &  mow    &  i     & n             &             &               &                \\
21\textsc{pl}    &  mow    &  i     &               & nân         &               &                 \\
2\textsc{pl}     &  mow    &  i     & k             &             &               &                  \\
\hline
\hline
2\textsc{sg}     &  mow    &  i     &               &             & hkan          &                   \\
21\textsc{pl}    &  mow    &  i     &               &             &               & hkâhk             \\
2\textsc{pl}     &  mow    &  i     &               &             & hkêk          &                     \\

    \bottomrule
  \end{tabular}
  \caption{
    VTA Imperative Local \citep[403]{Wolvengrey2011}. \label{tab:vtaimplocal}
  }
\end{table*}

The Imperative paradigms for the VTAs looks somewhat different than the VAI and VTI paradigms. Because the VTAs take two animate participants, the Imperative paradigm includes both first person and third person goals, as seen in Tables \ref{tab:vtaimpmixed} and \ref{tab:vtaimplocal}.


All forms except 2\textsc{sg} and 2\textsc{pl} acting on third persons in the Immediate imperative have a theme morph, \{-â-\} for the Mixed Participant Paradigm and \{-i-\} for the local. As in other cases, where one morpheme ends with /w/ and another begins with /i/, the surface form is realized as /o/, as in the Immediate second person acting on third singular for \textit{wâpamihkok}, `witness him, y'all!'

\subsubsection{Morphology Summarized}
Morphologically, and in particular from a structural point of view, it is obvious that the Independent and the Conjunct have similar paradigmatic shapes: they each mark for the same persons and make use of similar prefixes (though the Conjunct does so more uniformly than the Independent) and suffixes to mark these persons. Conversely, the Imperative exhibits a far more restricted paradigm: among actors it marks only for second person and makes no use of person prefixes. Further, while the Independent and the Conjunct can occur in any verb class, the Imperative and VIIs are mutually exclusive. These factors, at least on their own, suggest an ontology that place the imperative separately from the Independent and Conjunct, which are more similar to each other. This is illustrated in Figure \ref{fig:ImpOther}.

\begin{figure}[h]
\centering
\Tree[.Order 
        [.Imperative ] 
        [.$\neg$Imperative 
            [.Independent ] 
            [.Conjunct ]
        ] 
    ]
\caption{Morphological Ontology}
\label{fig:ImpOther}
\end{figure}

As will be seen throughout the rest of this chapter, this pattern of two orders being similar while the remaining one stands apart is pervasive through various levels of representation. This poses difficulty for creating a description or analysis of Order as a unified tripartite system, as one order seems to act substantially different from the others.

\subsection{Syntax}

Progressing from Morphology, I will now discuss the syntax of the three canonical Nêhiyawêwin Orders. The syntactic differences between the Independent, Conjunct, and the Imperative orders are best described by \citet{Cook2014}. Although \citet{Wolfart1973} touches on these differences, he does so without great detail. \citet{Wolfart1973} mentions that while the Imperative and the Independent can stand alone (without a prior clause or referent), the Conjunct often represents some form of subordination (which requires another clause on which to depend). Further, he describes each of his four kinds of Conjunct forms as follows: the Simple Conjunct (without IC or a subjunctive suffix) generally follows future markers or conjunctions such as \textit{nawac}, `should', or \textit{pitanê}, `would that/may'; conversely, the Changed Conjunct (with IC but not a subjunctive suffix) indicates subordination with little other syntactic restrictions; The Iterative Conjunct (with both IC and the subjunctive suffix) generally occurs in narrative and participial clauses, and finally, the Subjunctive Conjunct (without IC but with a subjunctive suffix) represents some sort of conditionally and often futurity \citep[46]{Wolfart1973}. Similarly, Cook details the syntactic distribution of the Conjunct order, explaining like \citet{Wolvengrey2011}, that the Conjunct can occur in subordinate (i.e. dependent clauses) (\citeyear{Cook2014}). In particular, Cook describes the Conjunct as \textit{mostly} occuring in these subordinate clauses, but with her Changed Conjunct$_{1}$ class as additionally being possible in matrix clauses. A summary of Cook's Conjunct subtype distinction is found in Table \ref{CookConj2} (\citeyear[125]{Cook2014}).

\begin{table}[h]
\centering
  \begin{tabular}{llllll}
\toprule
Submode   & Subtype     & Form       & Matrix & Subordinate                      \\
\midrule
Changed   & Changed Conjunct$_{1}$    & \^{e}-apiy\^{a}n   & \checkmark     & \checkmark \\
          & Changed Conjunct$_{2}$    & k\^{a}-apiy\^{a}n  & \ding{55}     & \checkmark  \\
          & Iterative                 & \^{e}piy\^{a}ni    & \ding{55}     & \checkmark   \\
Unchanged & Simple                    & ka-apiy\^{a}n      & \ding{55}     & \checkmark   \\
          & Subjunctive               & apiy\^{a}ni        & \ding{55}     & \checkmark    \\
\bottomrule
  \end{tabular}
  \caption{
    Description of Conjunct Orders (adapted from \citet[125]{Cook2014})}
     \label{CookConj2}
\end{table}

Although Cook explicitly does not discuss the Imperative, its syntactic distribution is like similar to that of the Independent. Cross linguistically, it has been reported that imperatives `tend not to occur as dependent clauses' \citep[174]{sadzw}. \citet{Wolfart1973} mentions that the imperative is often, but not exclusively, used along side a conditional clause, but in his examples, he gives only instances where the imperative verb is used in a matrix clause that contains a conditional subordinate clause. Alternatively, \citet[476]{Lakoff1984PerformativeSC} contends that Imperatives \textit{can} occur in subordinate clauses provided the subordinate be introduced by \textit{because} and the imperative actually convey a statement rather than an order. It is worth noting, however, that the evidence is provided for English, are not based in corpora or acceptability-judgement studies, and that the resulting `grammatical' sentences (e.g. \textit{I'm staying because consider the girl who pinched me}) are almost categorically ungrammatical to my ear. \citet{takahashi2008imperatives} presents a different approach, arguing that, at least in English, imperatives may be used as commands in certain concessive subordinate classes (e.g. \textit{I am going to Toronto, although don't expect me to bring you anything back!}). Little has been written about this phenomenon in Nêhiywawêwin, and to do so would be beyond the scope of this dissertation. What can be said is that the Imperative is not \textit{exclusively} used in embedded clauses. This results in two organizational structures. The first patterns the Imperative syntactically with the Independent and the Changed Conjunct$_{1}$ as all three are restricted to matrix clauses, as in Figure \ref{fig:ImpSyn1}.

\begin{figure}[h]
\centering
\resizebox{\textwidth}{!}{%
\Tree[.Order 
        [.Matrix 
            [.Independent ]
            [.{Changed Conjunct$_{1}$} ]
            [.Imperative ]
        ] 
        [.Embedded 
            [.Changed 
                [.{Changed Conjunct$_{1}$} ] 
                [.{Changed Conjunct$_{2}$} ] 
                [.Iterative ]
            ]
            [.Unchanged Simple Subjunctive ]
        ] 
    ]
    }
\caption{Syntactic Ontology 1}
\label{fig:ImpSyn1}
\end{figure}

The second possibility is one where the Imperative occurs both in both Matrix and Embedded clauses, as in Figure \ref{fig:ImpSyn2}. In either of these situations, the syntactic system does not cleanly align with the morphological system of Order.

\begin{figure}[h]
\centering
\resizebox{\textwidth}{!}{%
\Tree[.Order 
        [.Matrix 
            [.Independent ]
            [.{Changed Conjunct$_{1}$} ]
            [.Imperative ]
        ] 
        [.Embedded 
            [.Imperative ]
            [.Changed
                [.{Changed Conjunct$_{2}$} ] 
                [.Iterative ]
            ]
            [.Unchanged Simple Subjunctive ]
        ] 
    ]
    }
\caption{Syntactic Ontology 2}
\label{fig:ImpSyn2}
\end{figure}

\subsection{Semantics and Pragmatics}
The semantics and pragmatics of Nêhiyawêwin Order can be broken down into two main theoretical constructs: (1) \textit{sentence typing}, and (2) \textit{clause typing}. Here, \textit{sentence typing} refers to the three `basic sentence types' as described by \citet{konig2007speech}, who identify the \textit{declarative}, the \textit{imperative}, and the \textit{interrogative} as widespread typological phenomenon. These three Sentence Types are also represented in Nêhiyawêwin. While the Imperative order obviously corresponds to the imperative sentence type, the Independent and the Conjunct do not each represent one of the remaining sentence types. Instead, both the Independent and the Conjunct are able to be used as declarative constructions (in an unmarked or elsewhere case) as well as interrogatives (by making use of the \{cî\} clitic). This produces an ontology similar to the morphological organization seen previously, demonstrated in Figure \ref{fig:sem}.

\begin{figure}[h]
\centering
\Tree[.Order 
        [.Imperative 
            [.Immediate ]
            [.Delayed ]
        ]
        [.Declarative/Interrogative 
            [.Independent ] 
            [.Conjunct ]
        ] 
    ]
\caption{Semantic Ontology}
\label{fig:sem}
\end{figure}

For \citet{Cook2014} the use of Order comes down to clause typing. Here, \citet{Cook2014} distinguishes between indexical and anaphoric clauses. Indexical clauses are those that are grounded to the speech act, as in (\ref{index}). Indexical clauses are evaluated according the speaker as well as the time and place of the speech act; on the other hand, anaphoric clauses are evaluated according some different anchor \citep{Cook2014}. 

\begin{exe}
\ex
\gll mistahi \textbf{kî-miyohtwâ-wak}                                 êkonik           ôk           âyisiyini-wak        kâ-kî-ohpikih-iko-yâhkik \\
     extremely \textbf{\textsc{pst}-be.kind.\textsc{vai}-3\textsc{pl}}  \textsc{dist.pl} \textsc{foc.pl} person-\textsc{pl}   \textsc{cnj-pst}-raise.\textsc{vta}-\textsc{inv.thm}-\textsc{3pl}.\textsc{1pl}                                \\
\trans `The people who raised us (...) \textbf{they were extremely good people.} \citep[38]{AhenakewAlice2000}'
\label{anconj}
\end{exe}


This is perhaps most clearly instantiated in the use of the \{kî-\} morph, which is used with past events. According to \citep[125]{Cook2014}, this past morph is interpreted in an unspecified way in Conjunct clauses, which Cook identifies as inherently anaphoric, but is interpreted with a strictly modal (and non-tense) meaning in the independent.  \citet{Cook2014} describes these anaphoric clauses as being licensed by some antecedent, present in the discourse or in the real world knowledge of the interlocutors. Essentially, \citet{Cook2014} describes anaphoric clauses as having \textit{some} sort of semantic or syntactic relation with a licensor in another clause (as in (\ref{anconj})). She also contends that, in Nêhiyawêwin, anaphoric clauses are an elsewhere case that are defaulted to when an indexical clause is not present. The non-iteratve subjunctive form is not included by Cook, and its placement remains unclear 

\begin{exe}
\ex
\gll mistahi kî-miyohtwâ-wak                                 êkonik           ôk           âyisiyini-wak        \textbf{kâ-kî-ohpikih-iko-yâhkik} \\
     extremely    \textsc{pst}-be.kind.\textsc{vai}-3\textsc{pl}  \textsc{dem.pl} \textsc{foc.pl} person-\textsc{pl}   \textbf{\textbf{\textsc{cnj-pst}-raise.\textsc{vta}-\textsc{inv.thm}-\textsc{3pl}.\textsc{1pl}}}                                \\
\trans `\textbf{The people who raised us} (...) they were extremely good people. \citep[38]{AhenakewAlice2000}'
\label{anconj}
\end{exe}


Focusing specifically on the Conjunct modes, \citep{Cook2014} distinguishes these forms by the ways in which their pragmatic/semantic propositions are introduced: the Changed Conjunct$_2$ and Iterative  presuppose propositions, while Changed Conjunct$_1$ do not. Like the Changed Conjunct$_1$ forms, simple Conjuncts were not presuppositions, but are distinguished from Changed Conjunct$_1$ forms in that the latter are veridical statements, while simple Conjuncts are averidical \citet[302]{Cook2014}.\footnote{It is unclear where Cook would place her subjunctive Conjunct in terms of veridicality, though given her placement of it as a type of `simple conjunct', it seems possible that it would be an averidical form} An adaptation of Cook's Order ontology is found in Figure \ref{fig:cookorder}. 

\begin{figure}[h]
\centering
\Tree[.Order\\{(Clause-Typing)} [.\textit{Independent}\\(Indexical) ] [.\textit{Conjunct}\\(Anaphoric) [.Non-presupposed \textit{{Changed Conjunct$_{1}$}}\\(Veridical) \textit{Simple}\\(Averidical) ] [.Presupposed \textit{{Changed Conjunct$_{2}$}} \textit{Iterative} ] ] ]
\caption{Order Ontology in \citet{Cook2014}}
\label{fig:cookorder}
\end{figure}


Cook does not include the Imperative Order in her study, and it is difficult to determine where it would be placed in her ontology. Broadly, the imperative is clearly a clause type of its own: it represents an imperative clause as distinguished from declarative and interrogatives. If an indexical clauses is one that is rooted in the speech act. The definition of \textit{indexical} provided could just as easily apply to the Imperative Order. Indeed, \citet[111]{alc2014} describes the Imperative (independent of any specific language) as "encoding the (indexical) parameters of the speech act, such as participant roles, temporality and locality". Under this analysis, we find the ontology found in Figure \ref{fig:extendorder}.

\begin{figure}[h]
\centering
\Tree[.Order\\{(Clause-Typing)} [.Indexical \textit{Independent} \textit{Imperative} ] [.\textit{Conjunct}\\(Anaphoric) [.Non-presupposed \textit{{Changed Conjunct$_{1}$}}\\(Veridical) \textit{Simple}\\(Averidical) ] [.Presupposed \textit{{Changed Conjunct$_{2}$}} \textit{Iterative} ] ] ]
\caption{Order Ontology in \citet{Cook2014}}
\label{fig:extendorder}
\end{figure}

 Regardless of these interpretations, this sort of classification of Order, like others, necessarily treats the Independent, Conjunct, and Imperative not of the same kind (as is done in traditional descriptions of Algonquian grammar), but position Conjunct as opposed to an Independent-Imperative conglomerate (distinct from descriptions by \citet{Bloomfield1930}, \citet{Wolfart1973}, \cite{Wolvengrey2011} and others, which group the Independent and Conjunct together as opposed to the Imperative). 
 
 
 \subsubsection{Conjunct modes in This Dissertation}
As shown, while both agree that modes of the Conjunct exist, \citet{Wolfart1973} and \citet{Cook2014} vary in their descriptions of them. In order to study Order, it is critical to operationalize what different modes exist. Rather than simply taking either the \citet{Wolfart1973} or \citet{Cook2014}, I opt to use corpus evidence to define the Conjunct modes on a structural basis.  In the subset of the Ahenakew-Wolfart corpus \citep{arppe1945morphosyntactically} used for this dissertation (see Chapter \ref{ch:method} for more detail), the following set of morphological patterns were found:

\begin{itemize}
    \item ê- Initial (6373 tokens)
    \item ka-/ta- Initial (910 tokens)
    \item kâ- Initial (2458 tokens)
    \item Initial Change (54 tokens)
    \item Subjunctive \{-i\} (172 tokens)
\end{itemize}

Interestingly, there were no forms in the analyzed corpus that contained both a subjunctive suffix \textit{and} IC (the \textit{iterative} in \citet{Wolfart1973} and \citet{Cook2014}). While the corpus lacked an iterative, it did contain verbs with \textit{only} IC,\footnote{This may be due, at least regarding IC, to the fact that \{ê-\} was historically nothing more than a vehicle to indicate Initial Change \citep[46]{Wolfart1973}. In this way, one could consider the \{ê-\} prefixed Conjuncts as inherently Changed, though synchronically this is non-obvious. As a result, the remainder of this dissertation will not consider the \{ê-\} prefixed Conjuncts as examples of Initial Change.} a form seemingly missing in \citet{Cook2014}. Further, the naming conventions used by \citet{Cook2014} and \citet{Wolfart1973} will not be used for this dissertation. Instead, I will refer to the Conjunct modes by their prefixes. The only exceptions to this are those forms where there is only initial change and those forms suffixed with the subjunctive morph. Because they can not be identified by a single prefix, they will be called the \textit{Initial Change Conjunct} and the \textit{Subjunctive Conjunct}.

In considering types of Conjunct, there is a structural difference between those types that have a grammatical, Conjunct specific, preverb such as \{ê-\}, \{ka-\}/\{ta-\}, and \{kâ-\}. These forms can be thought of as being \textit{prefixed}, while the Initial Change and Subjunctive forms can be considered \textit{bare}, due to their lack of a Conjunct prefix. Both Initial Change and Subjunctive forms have only a small number of tokens. Bare tokens with Conjunct endings but lacking either the Subjunctive \{-i\} or IC were excluded as contemporary speakers considered them as 'incorrect,' and their frequency in the corpus was even smaller than that of the Initial Change Conjunct.


\begin{figure}[h]
\centering
\Tree[.Order [.Imperative ] [.$\neg$Imperative [.Independent ] [.Conjunct [.Prefixed ê- kâ- ka- ] [.Bare IC Subjunctive ] ] ]  ]
\caption{Morphological Ontology of Order}
\label{fig:struct}
\end{figure}



While morphologically the Imperative Order is similar to the Conjunct and Independent Orders, its inability to take all arguments as well as the conjunct or Independent person-marking preverbs sets the Imperative apart. In fact, the most salient similarity between the Imperative, the Independent, and the Conjunct orders is that all three are able to inflect for second person items, at least in the non-VII classes. Beyond this, the Independent and Conjunct Orders can take exactly the same persons, but differ in the exponents used. This results in a morphological system as visualised in Figure \ref{fig:struct}. Although the corpus used in this dissertation does not include Iterative Conjuncts, one could include them as a type of Subjunctive Conjunct (as both contain the Subjunctive suffix), resulting in the structure of Figure \ref{fig:struct2}.

\begin{figure}[h]
\centering
\Tree[.Sentence-Type [.Imperative [.Immediate ] [.Delayed ]] [.Order [.Independent ] [.Conjunct [.Prefixed ê- kâ- ka- ] [.Bare [.IC IC Iterative ] [.$\neg$IC Subjunctive ] ] ]  ] ]
\caption{Morphological Ontology of Order 2}
\label{fig:struct2}
\end{figure}

Alternatively, one could group the iterative with the Subjunctive, thus creating a Bare distinction between item with an \{-i\} suffix and those without, as in Figure \ref{fig:struct3}.

\begin{figure}[h]
\centering
\Tree[.Sentence-Type [.Imperative [.Immediate ] [.Delayed ]] [.Order [.Independent ] [.Conjunct [.Prefixed ê- kâ- ka- ] [.Bare [.$\neg$\{-i\} IC ] [.\{-i\} Subjunctive Iterative ] ] ]  ] ]
\caption{Morphological Ontology of Order 3}
\label{fig:struct3}
\end{figure}

There is no good theoretical reason to chose one of these options over the other. One could also choose to treat the Subjunctive, IC, and Iterative conjuncts as three separate nodes, grouping none together. While this seems as valid as the previous two ontologies, it ignores the similarities of these classes. In fact, because all bare forms are combine fore the sake of analysis in this dissertation, this distinction is not material for this dissertation.

\FloatBarrier 

\subsection{Summary of Order}
Nêhiyawêwin Order has been described as a system of linguistic features cross cutting various levels of representation. Morphologically, Order is a structural phenomenon where Algonquian language use various exponents to mark person on verbs. Under this definition, we can identify three Orders: 
\begin{enumerate}
    \item Those where the VAI, VTI and VTA classes use circumfixes with \{ni-\} prefixes for first person and \{ki-\} prefixes for second person (the Independent)
    \item Those with the prefixes \{ê-\}, \{ka-\}/\{ta-\}, \{kâ-\}, or Initial Change regardless of person (the Conjunct)
    \item Those which use neither of these strategies (the Imperative)
\end{enumerate}

This places the Independent and Conjunct together against the Imperative (which is essentially defined as not being Independent or Conjunct). Alternatively, we can identify two Orders:

\begin{enumerate}
    \item Those that can mark for first, second, third, and obviative persons (the Independent and Conjunct)
    \item Those that can mark only for the second person (the Imperative)
\end{enumerate}

Again, in this situation the first of these proposed Orders would include what is traditionally called the Independent \textit{and} what is traditionally called the Conjunct, with the second class making up the Imperative.

If we choose to define the phenomenon in terms of semantic, syntactic, and pragmatic behaviour, we can refer to Figure \ref{fig:extendorder}, wherein Independent and Imperative are indexical, while the Conjunct in anaphoric. Contrary to the previous descriptions, this places Conjunct apart from the other Orders.

\begin{figure}[h]
\centering
\Tree[.Order\\{(Clause-Typing)} [.Indexical \textit{Independent} \textit{Imperative} ] [.\textit{Conjunct}\\(Anaphoric) [.Non-presupposed \textit{{Changed Conjunct$_{1}$}}\\(Veridical) \textit{Simple}\\(Averidical) ] [.Presupposed \textit{{Changed Conjunct$_{2}$}} \textit{Iterative} ] ] ]
\repeatcaption{fig:extendorder}{Order Ontology in \citet{Cook2014}}
\end{figure}


Finally, if we consider Order purely in terms of semantics, we can define Order as system of distinguishing mood (the imperative vs. the declarative). In this classification, the Independent and Conjunct are not distinguished by mood in the same way that they can be contrasted against the Imperative (cf. \ref{fig:semantics} ).

\begin{figure}[h]
\centering
\Tree[.Order [.Mood Imperative ] [.$\neg$Mood Independent Conjunct ] ]
\caption{Semantic Order Ontology}
\label{fig:semantics}
\end{figure}

Thus,  we again have a situation where the Imperative is of a different kind than the Independent/Conjunct. Regardless of what scheme one uses to describe Order in Nêhiyawêwin, there is no way to divide the Independent, Imperative, and Conjunct such that they're all of the same kind or on the same level. The best argument for equating the Independent, Conjunct, and Imperative is that occurrence in one of these precludes occurrence in the other (i.e. there is no such thing as an Independent Imperative).\footnote{Unless, of course, one identifies the Delayed Imperative as a Conjunct form for the Imperative, as described above.} Under this definition, a tri-partite Order is essentially an operation that takes a verb stem, a linguistic person or persons, and a direction (if needed) and produces a surface form as in (\ref{funct}).\footnote{I make no claim of psychological reality in this statement, it is purely metaphorical.}

\begin{exe}
 \ex \mbox{Conjunct}(\mbox{wâpam},(\mbox{1\textsc{sg}},\mbox{3\textsc{sg}}),\mbox{inverse}) = \mbox{ê-wâpamit}
 \label{funct}
\end{exe}

In this way, one can think of Order as an operation that applies to verbs; however, the Imperative is incompatible with the VII class, while both the Independent and Conjunct can apply to any class. Even considering Order as this sort of formal function leads to a distinction between the Imperative and the Independent/Conjunct. In terms of structure, behaviour, and semantics, this difference persists. This conflict is problematic to the study of Nêhiyawêwin grammar, as any claim about Order needs to be relevant to all three of these categories. For these reasons, the use of the term \textit{Order} will be redefined in this dissertation. Instead of creating a three way split between Independent, Conjunct, and Imperative, I will consider Order to be a grouping of allomorphic alternations in the paradigm. Therefore, in this dissertation will refer only to the Independent and Conjunct. 

In terms of describing what the Imperative is, if not Order, I propose that the Imperative is a construction that acts as an illocutionary force indicating device \citep{searle1985speech} marking a command. Under this system, we can understand the interrogative to be marked through the use of the \{cî\} morph, and the declarative to remain unmarked. Thus the concept of mood (which is mostly imparted by preverbs in Nêhiyawêwin) is made separate from the idea of Order entirely. Thus, while the Independent and Conjunct may still be referred to as Order, the Imperative is of the sentence-type or illocutionary force. 

\subsection{Alternation}

By extricating the Imperative from the system of order, we are left with a binary distinction of Independent and Conjunct. This juxtaposition presents two ways of encoding person number with different morphemes. In other words, while the shape and grammatical content (e.g. both Orders mark exactly the same persons) of the \textit{paradigms} are the same, the actual exponents that are realized in the cells are not. According to \citet{Cook2014}, these two alternatives correspond one-to-one to clause typings, the indexical and the anaphoric. This view of Order as two alternative constructions used to encode different meaning is essentially one of \textit{alternation}.

In its broadest conception, the idea of an alternation is simply one in which some linguistic form—be it phonological, morphological, syntactic, or other—is contrasted with another. Perhaps most popular are the alternation of allophony and allomorphy, wherein two or more forms are used interchangeably according to some context. Sociolinguistic studies, such as Labov’s famous study of English retroflex consonants in New York City department stores (\citeyear{labov1986social}), often make use of the concept of the alternation by framing the realization of a phoneme as one of a set of allophones that change depending on some contextual criteria. In addition to this basic sociolinguistics-based approach, there exists a constructional/psycholinguistic based approach to alternations. \citet{pijpops2020} presents an overview of the concept covering the three traditional definitions (1-3) along with three more recently developed conceptions (4-6):
\begin{enumerate}
 	\item Alternations share the same meaning, are similarly processed in the mind, and vary according to some dialectal factor.
\item Alternations share the same meaning, do \textbf{not} vary according to dialect, but \textbf{are} differently processed in the mind.
\item Alternations have a difference in meaning that varies due to some lexical influence
\item Alternations represent any point where the speaker must make a choice in what is said
\item Alternations are a tool to analyze phenomenon that a linguist deems interesting
\item Alternations are items with special theoretical relations to one another
\end{enumerate}

Collectively, these definitions privilege the sociolinguistic/psycholinguistic/processing aspects. While they consider the concept of meaning similarity, they define alternation primarily by factors such as dialect, psycholinguistic processing, or as some sort of linguistic tool of convenience. In fact, \citet{pijpops2020} suggests this latter definition when citing \citet{arppe2010cognitive}. However, the claim that \citet{pijpops2020} cites from \citet{arppe2010cognitive} comes from only one of the five authors; others, such as Arppe and Gilquin, instead argue that alternations do, in fact, exist and can be valid areas of study in their own right. Additionally, the criticism of alternations by Glynn appears to be based upon: (1) the idea that alternation studies are an artifact of Cognitive and Construction Grammars responding to Generative Grammars, and (2) that alternations are inherently non-binary and researchers lack the knowledge of multinomial statistics. The latter of these claims is something that may have been true in 2010\footnote{It wasn't.}, but is certainly not true today. The former of these claims, even if it were true, does not comment on the validity of the idea of an alternation as a subject of study. Being originally a response to generative grammar is not in and of itself a problem for alternation studies.

 In addition to \citet{pijpops2020}'s definitions, there are other ways to approach alternations. Specifically, one can make use of a lexicographically grounded approach which considers the concept of synonymy and the way in which synonyms and near synonyms can be used in similar (but not identical) contexts. In this vein, \cite[156]{cruse2001} discusses the concept of synonymy, which he defines as not simply as words with the same meaning, but “words whose semantic similarities are more salient than their differences.” In particular, Cruse identifies three types of synonyms: absolute synonyms (which are fully equivalent in all cases and occur rarely, if at all), propositional synonyms, (which may alternate without changing the truth condition of a statement, but which may differ in speaker attitude or register), and pleisonyms/near-synonyms (which can be said to share core semantic properties, even if they differ in ‘minor’ or ‘background’ ways) \citep[157-159]{cruse2001}. Because any of these forms of synonymy necessarily concern the employment of one of many forms for the same referent, synonymy is a straightforward and clear case of alternation. 

Following from this lexicographic approach, alternations can be construed on various levels: conceptual-semantic alternation, stylistic-semantic alternation, and a syntactic-semantic alternation (\citealt[8]{arppe2008univariate};  cf. \citealt{edmonds2002near} for an earlier discussion of a similar concept). According to \citet[8]{arppe2008univariate}, conceptual-semantic alternations concern words that mean generally the same thing and can be used (roughly) interchangeably (e.g. \textit{dash} and \textit{sprint}); Stylistic-semantic alternations occur between words or phrases that share similar meanings, but contain different connotations (\textit{poop} and \textit{shit}); and syntactic-semantic alternations deal with similar utterances which take different syntactic patterns (\textit{comb (through)} and \textit{inspect}). These levels of representation consider alternations as near-synonymous pairs that can make use of three latter definitions presented by \citet{pijpops2020}, particularly as a point-of-choice. They also roughly correspond to those of Hanks’ lexical, semantic, and syntactic-type alternations (\citeyear[173]{hanks2013lexical}). \citet[10]{arppe2008univariate} also proposes a subset of syntactic-semantic alternations referred to as constructional alternations, which keep the same central meaning, though which may differ in more subtle, often pragmatic dimensions. Framing a phenomenon as an alternation creates a structured difference that researchers can investigate. The previous sections have demonstrated the ways in which Nêhiyawêwin Order, at least as conceived as a tripartite system can not be thought of as an alternation. The decision in which one would decided to use the Imperative (that is, the choice of sentence typing) is not the same decision as between the Independent and Conjunct. Thus, while alternation is not a useful tool for the study of the morphological difference between the Independent, Conjunct, and Imperative, it may be useful for the the difference between the Independent and Conjunct. 

I argue that the phenomenon of binary Order, between Independent and Conjunct, is a form of nearly synonymous constructional alternation, but one that has remained, as of yet, undescribed. I propose that order represents a \textit{paradigmatic alternation}. A paradigmatic alternation is here defined as one where \textit{any} lexeme of a particular word class is able to take two or more different paradigms but where each of those paradigms is identical in shape but different in exponence. This differs from similar phenomena such as noun class. In such a phenomenon, there are indeed alternating paradigms, for example Latin's masculine, feminine and neuter, with similar or identical shapes with differing exponents. Here however, it is not the case that any noun can occur in any paradigm. Instead, the paradigm which a lexeme occurs in is functionally a an attribute of a lexeme.

Viewed as an analysis of an alternation, the primary research question of this dissertation is as follows: what morphosyntactic and semantic features affect a lemma’s propensity to occur in a particular alternation of Order or mode. Adopting a usage-based approach based in the distributional hypothesis \citep{harris1954distributional, firth1957synopsis}, this research will utilize quantitative methodologies in an effort to see to what extent empirical, corpus-based evidence can guide us in the understanding of Nêhiyawêwin order.
